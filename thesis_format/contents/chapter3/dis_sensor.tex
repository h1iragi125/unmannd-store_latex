\graphicspath{{./contents/chapter3/figures/}}	% 図・画像ファイルが保存されたディレクトリまでのパス
\lstset{inputpath=./contents/appendix/source_code/program_files/}

\chapter{超音波センサによる在庫の把握}
	本章では、本研究で行った超音波センサによる在庫把握の手段について述べる。
	\section{超音波センサによる在庫把握の方法}
		本研究では、無人店舗内の商品棚における在庫の把握を超音波センサを用いることによって行った。商品棚の奥側に超音波センサを設置し、センサから発信された超音波が商品に反射して戻ってくるまでの時間を距離に計算し直すことで行う。距離は以下の式で求められる。
		\begin{equation}
			距離 = \frac{音速 \times 時間}{2}
		\end{equation}
		ここで、音速は約343m/sである。計測した距離の長短で在庫の状況を判断する。使用した超音波センサは\figref{ultrasound_sensor}に示すRainbow E-Technology社のHC-SR04である。このセンサは単体では動作しないため、マイコンボードと組み合わせて使用する必要がある。超音波センサで取得したデータをUSB経由でパソコンに送信し、そのデータをパソコンで評価する。

		\vspace{0.5\baselineskip}
		\begin{figure}[h]
			\centering
			\includegraphics{hcsr04.jpg}
			\caption{Rainbow E-Technology社 HC-SR04}
			\label{ultrasound_sensor}
		\end{figure}
	
	\section{研究に使用した装置}
		\subsection{超音波センサの性能}
			HC-SR04の性能を\tabref{HC-SR04}に示す。測定可能距離は0.02m~4.5mであり、商品棚での在庫把握は可能であると考えられる。
			\vspace{0.5\baselineskip}
			\begin{table}[h]
				\centering
				\caption{HC-SR04の精度}
				\begin{tabular}{|c|r|}
					\hline	動作電圧	& 3~5.5V \\
					\hline	測定可能距離	& 0.02~4.5m\\
					\hline	測定方式	& 超音波 \\
					\hline	動作温度	& -10~70℃\\
					\hline
				\end{tabular}
				\label{HC-SR04}
			\end{table}
	
		\subsection{マイコンボードを含めた装置全体}
			前述の通り、この超音波センサは単体では動作しないためマイコンボードと組み合わせる必要がある。マイコンボードにはRaspberry Pi Pico 2 WHを使用した。装置の全体図を\figref{souchi}に示す。
			\vspace{0.5\baselineskip}
			\begin{figure}[h]
				\centering
				\includegraphics[scale=0.1]{souchi1.jpg}
				\caption{装置全体図}
				\label{souchi}
			\end{figure}

	\section{超音波センサの配置}
		超音波センサを商品棚の奥側のスペースに配置し、商品との距離を計測する。


	\section{超音波センサの距離測定に使用するプログラム}
		超音波センサで距離を測定するにあたり、ソースコード\ref{measurepy}に示すPythonのプログラムをマイコンボードに書き込んだ。
		\hypertarget{measurepy}{
			\lstinputlisting[style=mypystyle,caption=マイコンボードに書き込んだPythonプログラム,label=measurepy]{measure.py}
		}
		センサで取得したデータをパソコンで評価するためのプログラムをソースコード\ref{receivepy}に示す。パソコンのOSはWindowsを想定している。
		\hypertarget{receivepy}{
			\lstinputlisting[style=mypystyle,caption=パソコンで実行するPythonプログラム,label=receivepy]{receive_windows.py}
		}

	\section{超音波センサの試行の結果と考察}
		