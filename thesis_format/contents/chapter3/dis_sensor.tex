\graphicspath{{./contents/chapter3/figures/}}	% 図・画像ファイルが保存されたディレクトリまでのパス
\lstset{inputpath=./contents/appendix/source_code/program_files/}

\chapter{超音波センサによる在庫の把握}
	本章では、本研究で行った超音波センサによる在庫把握の手段について述べる。
	\section{超音波センサによる在庫把握の方法}
		本研究では、無人店舗内の商品棚における在庫の把握を超音波センサを用いることによって行った。商品棚の奥側に超音波センサを設置し、センサから発信された超音波が商品に反射して戻ってくるまでの時間を距離に計算し直すことで行う。距離は以下の式で求められる。
		\begin{equation}
			距離 = \frac{音速 \times 時間}{2}
		\end{equation}
		ここで、音速は約343m/sである。計測した距離の長短で在庫の状況を判断する。使用した超音波センサは\figref{ultrasound_sensor}に示すRainbow E-Technology社のHC-SR04である。このセンサは単体では動作しないため、マイコンボードと組み合わせて使用する必要がある。超音波センサで取得したデータをUSB経由でパソコンに送信し、そのデータをパソコンで評価する。

		\vspace{0.5\baselineskip}
		\begin{figure}[h]
			\centering
			\includegraphics{hcsr04.jpg}
			\caption{Rainbow E-Technology社 HC-SR04}
			\label{ultrasound_sensor}
		\end{figure}
	
	\section{研究に使用した装置}
		\subsection{超音波センサの性能}
			HC-SR04の性能を\tabref{HC-SR04}に示す。測定可能距離は0.02m~4.5mであり、商品棚での在庫把握は可能であると考えられる。

			\vspace{0.5\baselineskip}
			\begin{table}[h]
				\centering
				\caption{HC-SR04の精度}
				\begin{tabular}{|c|r|}
					\hline	動作電圧	& 3~5.5V \\
					\hline	測定可能距離	& 0.02~4.5m\\
					\hline	測定方式	& 超音波 \\
					\hline	動作温度	& -10~70℃\\
					\hline
				\end{tabular}
				\label{HC-SR04}
			\end{table}
	
		\subsection{マイコンボードを含めた装置全体}
			前述の通り、この超音波センサは単体では動作しないためマイコンボードと組み合わせる必要がある。マイコンボードにはRaspberry Pi Pico 2 WHを使用した。装置の全体図を\figref{souchi}に示す。

			\vspace{0.5\baselineskip}
			\begin{figure}[h]
				\centering
				\includegraphics[scale=0.1]{souchi1.jpg}
				\caption{装置全体図}
				\label{souchi}
			\end{figure}

	\section{超音波センサの距離測定に使用するプログラム}
		超音波センサで距離を測定するにあたり、ソースコード\ref{measurepy}に示すPythonのプログラムをマイコンボードに書き込んだ。
		\hypertarget{measurepy}{
			\lstinputlisting[style=mypystyle,caption=マイコンボードに書き込んだPythonプログラム,label=measurepy]{measure.py}
		}
		センサで取得した距離から在庫を評価するためのプログラムをソースコード\ref{receivepy}に示す。パソコンのOSはWindowsを想定している。
		\hypertarget{receivepy}{
			\lstinputlisting[style=mypystyle,caption=在庫評価用のPythonプログラム,label=receivepy]{receive_windows.py}
		}
		評価の対象はベーグルとした。このプログラムでは在庫の状況を3段階で評価しているが、評価の基準はセンサから物体までの長さであるため、評価の対象を変える場合、プログラム上の数字も変える必要がある。

	\section{超音波センサの配置}
		超音波センサを商品棚の奥側のスペースに配置し、商品との距離を計測する。奥側に設置したのは、配線やセンサ本体が利用客の邪魔にならないようにするためである。

		ここに画像を貼る



	\section{超音波センサに関する課題}
		まず、無人店舗で取り扱っている商品は60種類とかなり多い。そのため、超音波センサを1商品に対し1台設置すると多額の費用が発生し、電源供給の課題も発生する。その課題を解決するには、1台のマイコンボードに対し2つ以上の超音波センサを設置するなどの工夫が必要となる。
		また、無人店舗において、店舗利用者が商品を手に取ったあと、商品棚の奥側にある在庫が前出しされないという問題がある。商品の前出しがされないと実際は在庫数が少ないが在庫数に余裕ありと判断されるため、在庫把握に超音波センサを用いる場合はこの問題を対処する必要がある。この問題を解決するために一部の小売店では\figref{facingstand}のような自動で前出しされるフェイシングスタンド\cite{3.1}を用いているが、無人店舗は商品の種類が多いため導入のハードルが高く、おにぎりのように潰れてしまいやすい商品があるため、導入は厳しい。

		\vspace{0.5\baselineskip}
			\begin{figure}[h]
				\centering
				\includegraphics[scale=0.5]{facingstand.jpg}
				\caption{フェイシングスタンド}
				\label{facingstand}
			\end{figure}