\graphicspath{{./contents/chapter4/figures/}}	% 図・画像ファイルが保存されたディレクトリまでのパス
\chapter{デプスカメラによる在庫把握}
	本章では、本研究で行ったデプスカメラによる在庫把握の手段について述べる。
	\section{デプスカメラによる在庫把握の方法}
		デプスカメラで無人店舗内の商品棚を撮影し、カメラと物体の距離を測定することで在庫把握を行う。
		使用したデプスカメラは\figref{b0410}に示すArducam社のB0410である。
	
		% \vspace{0.5\baselineskip}
		% \begin{figure}[h]
		% 	\centering
		% 	\includegraphics{SEN0581.jpg}
		% 	\caption{DFRobot社 SEN0581}
		% 	\label{sen0581}
		% \end{figure}

		\vspace{0.5\baselineskip}
		\begin{figure}[h]
			\centering
			\includegraphics{b0410.jpg}
			\caption{Arducam社 B0410}
			\label{b0410}
		\end{figure}

			\subsection{デプスカメラの精度}
				B0410の性能を\tabref{depth_spec}に示す。
				\vspace{0.5\baselineskip}
				\begin{table}[h]
					\centering
					\caption{デプスカメラの性能}
					\begin{tabular}{|c|r|}
						\hline  型番  & B0410\\
						\hline	解像度	& 240x180\\
						\hline	コマ数	& 30fps\\
						\hline	計測範囲	& 2mまたは4m\\
						\hline	画角	&  対角70°\\
						\hline	接続方法	& Raspberry Pi本体に接続\\
						\hline
					\end{tabular}
					\label{depth_spec}
				\end{table}

	\section{デプスカメラでの距離測定に使用するプログラム}
		デプスカメラで距離を測定するにあたり、Arducam社のSDK\cite{4.1}からRaspberry Piで実行するPythonのプログラムをダウンロードし、使用した。このプログラムを実行すると、デプスカメラで撮影された映像がRaspberry Pi上に表示される。



	\section{デプスカメラの配置と距離の検出方法}
		デプスカメラの配置方法は、防犯カメラのように店舗の壁面に設置する方法または超音波センサと同じように商品棚の奥側に設置する方法の2通りがある。前者の設置方法では1台のカメラで複数の商品の在庫を把握できるという利点があるが、今回使用したデプスカメラは解像度が非常に低いうえ、商品棚全体を認識できるようにデプスカメラを設置することが困難であった。そのため、本研究では後者の商品棚の奥側に設置する方法を採用した。

	\section{デプスカメラでの在庫把握に関する課題}
		デプスカメラの設置方法について本研究では商品棚の奥側に設置する方法を採用したが、この方法ではデプスカメラを大量に用意する必要があり、そのうえ電源供給の課題が発生する。また、今回使用したB0410は((()))ため、第2章で用いたHSV変換による物体検出の手法が使えない。


		商品棚全体を認識できるようにデプスカメラを配置できれば、Intel社のIntel RealSense Depth Camera D435などの高解像度かつカラー表示ができるカメラを用いることで、これらの問題が解決する。
		