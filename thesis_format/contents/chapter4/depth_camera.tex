\graphicspath{{./contents/chapter4/figures/}}	% 図・画像ファイルが保存されたディレクトリまでのパス
\chapter{デプスカメラによる在庫把握}
	本章では、本研究で行ったデプスカメラによる在庫把握の手段について述べる。
	\section{デプスカメラによる在庫把握の方法}
		デプスカメラで無人店舗内の商品棚を撮影し、カメラと物体の距離を測定することで在庫把握を行う。
		使用したデプスカメラは\figref{b0410}に示すArducam社のB0410である。このデプスカメラは3D-ToF(Time of Flight)方式のカメラで、赤外光を用いて距離を計測している。赤外光を用いることで、光が入らない暗い場所でも使用できる。

		\vspace{0.5\baselineskip}
		\begin{figure}[h]
			\centering
			\includegraphics{b0410.jpg}
			\caption{Arducam社 B0410}
			\label{b0410}
		\end{figure}

		\vspace{0.5\baselineskip}
		\begin{figure}[h]
			\centering
			\includegraphics[scale=0.05]{raspi.jpg}
			\caption{B0410を接続したRaspberry Pi}
			\label{raspi}
		\end{figure}

			\subsection{デプスカメラの精度}
				B0410の性能を\tabref{depth_spec}に示す。
				\vspace{0.5\baselineskip}
				\begin{table}[h]
					\centering
					\caption{デプスカメラの性能}
					\begin{tabular}{|c|r|}
						\hline  型番  & B0410\\
						\hline	解像度	& 240x180\\
						\hline	コマ数	& 30fps\\
						\hline	計測範囲	& 2mまたは4m\\
						\hline	画角	&  対角70°\\
						\hline	接続方法	& Raspberry Pi本体に接続\\
						\hline
					\end{tabular}
					\label{depth_spec}
				\end{table}
	
	\section{デプスカメラの配置}
		デプスカメラの配置方法は、防犯カメラのように店舗の壁面に設置する方法または超音波センサと同じように商品棚の奥側に設置する方法の2通りがある。前者の設置方法では1台のカメラで複数の商品の在庫を把握できるという利点があるが、今回使用したデプスカメラは解像度が非常に低いうえ、商品棚全体を認識できるようにデプスカメラを設置することが困難であった。そのため、本研究では後者の商品棚の奥側に設置する方法を採用した。

	\section{デプスカメラでの距離測定に使用するプログラム}
		デプスカメラで距離を測定するにあたり、Arducam社のSDKをダウンロードし、Raspberry Piで実行するPythonのプログラム(preview\_depth.py)を使用した。このプログラムを実行すると、デプスカメラで撮影された映像が\figref{fig:raspigazou}のようにRaspberry Pi上に表示される。ここで、画像左側の「preview\_confidence」は赤外線画像で、画面右側の「preview」は、計測した距離を並べて画像にしたものをOpenCVのカラーマップであるcv2.COLORMAP\_RAINBOWで擬似的にカラー化した画像である。
		
		\vspace{0.5\baselineskip}
				\begin{figure}[h]
					\centering
					\includegraphics[scale=0.3]{20260129_20h18m47s_grim.png}
					\caption{Raspberry Piのスクリーンショット}
					\label{raspigazou}
				\end{figure}
	

	\section{考察}
		デプスカメラを用いて物体とカメラ間の距離を計測することで、超音波センサでの課題を解決しながら商品の在庫把握が可能であると考えたが、本研究で採用したデプスカメラの設置方法では、デプスカメラを大量に用意する必要がありそのうえ電源供給の課題が発生する。B0410は1台あたり9000円でありカメラ1台につきRaspberry Piが1台必要なため、商品1種類につきB0410を1台用意する場合には多額の費用が発生する。

		商品棚全体を認識できるようにデプスカメラを配置できれば、カメラの台数を少なくしつつ第2章で用いたOpenCVによる画像処理で在庫把握が行える。だが、今回使用したB0410はカラー画像の出力には対応していないため、第2章で用いたHSV変換による物体検出の手法が使えない。
		
		高解像度かつカラー表示ができるデプスカメラを壁面に設置することで、これらの問題が解決するが、依然として商品棚に陳列されている在庫の具体的な個数の把握は困難である。