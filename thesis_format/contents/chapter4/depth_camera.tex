\graphicspath{{./contents/chapter4/figures/}}	% 図・画像ファイルが保存されたディレクトリまでのパス
\chapter{デプスカメラによるデータ収集}
	\section{使用するデプスカメラの説明}
		本研究ではデプスカメラを用い、センサと商品間の距離を計測した。使用したセンサはである。このセンサ単体では動作しないため、マイコンボードと組み合わせて使用する必要がある。マイコンボードにはRaspberry Pi 4 Model Bを使用した。
	
			\subsection{デプスカメラの精度}
				D435の性能をに示す。測定範囲は0.1m~10mであり、混雑状況のモニタリングの際に観測される距離は測定可能であると考えられる。
				
				\vspace{0.5\baselineskip}
				\begin{table}[h]
					\centering
					\caption{D435の精度}
					\begin{tabular}{|c|r|}
						\hline	動作電圧	& 5V \\
						\hline	消費電流	&	150mA\\
						\hline	視野	&  87°×58°\\
						\hline	測定範囲	&	0.1m~10m\\
						\hline	深度分解能	& 1mm\\
						\hline
					\end{tabular}
					
				\end{table}
	\section{デプスカメラの配置と距離の検出方法}
		デプスカメラを11号館1階の無人店舗の商品棚の奥側のスペースに配置し、商品との距離を計測する。距離の検出方法は、カメラから赤外線を発信し、商品に反射して戻ってくるまでの時間を計測することで行う。距離は以下の式で求められる。
