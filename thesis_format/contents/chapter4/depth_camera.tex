\graphicspath{{./contents/chapter4/figures/}}	% 図・画像ファイルが保存されたディレクトリまでのパス
\chapter{デプスカメラによるデータ収集}
	本章では、本研究で行ったデプスカメラによる在庫把握の手段について述べる。
	\section{デプスカメラによる在庫把握の方法}
		本研究では、超音波センサだけではなくデプスカメラも用いて無人店舗内の在庫数の把握を行った。使用したデプスカメラは\figref{sen0581}に示すDFRobot社のSEN0581と\figref{b0410}Arducam社のB0410である。
	
		\vspace{0.5\baselineskip}
		\begin{figure}[h]
			\centering
			\includegraphics{SEN0581.jpg}
			\caption{DFRobot社 SEN0581}
			\label{sen0581}
		\end{figure}

		\vspace{0.5\baselineskip}
		\begin{figure}[h]
			\centering
			\includegraphics{b0410.jpg}
			\caption{Arducam社 B0410}
			\label{b0410}
		\end{figure}

			\subsection{デプスカメラの精度}
				これらのデプスカメラの性能を\tabref{depth_spec}に示す。
				\vspace{0.5\baselineskip}
				\begin{table}[h]
					\centering
					\caption{D435の精度}
					\begin{tabular}{|c|r|r|}
						\hline	解像度	& 240x180 & 100x100 \\
						\hline	コマ数	& 30fps & 1~20fps\\
						\hline	計測範囲	&  対角70° & ×58°\\
						\hline	画角	&	0.1m~10m\\
						\hline	接続方法	& 1mm\\
						\hline
					\end{tabular}
					\label{depth_spec}
				\end{table}
	\section{デプスカメラの配置と距離の検出方法}
		