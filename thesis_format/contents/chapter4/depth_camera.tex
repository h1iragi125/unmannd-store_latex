\graphicspath{{./contents/chapter4/figures/}}	% 図・画像ファイルが保存されたディレクトリまでのパス
<<<<<<< HEAD
\chapter{デプスカメラによる在庫検出}
=======
\chapter{デプスカメラによるデータ収集}
	本章では、本研究で行ったデプスカメラによる在庫把握の手段について述べる。
	\section{デプスカメラによる在庫把握の方法}
		本研究では、超音波センサだけではなくデプスカメラも用いて無人店舗内の在庫数の把握を行った。使用したデプスカメラはDFRobot社のSEN0581とArducam社のB0410である。
	
		\vspace{0.5\baselineskip}
		\begin{figure}[h]
			\centering
			\includegraphics{SEN0581.jpg}
			\caption{DFRobot社 SEN0581}
			\label{sen0581}
		\end{figure}

		\vspace{0.5\baselineskip}
		\begin{figure}[h]
			\centering
			\includegraphics{SEN0581.jpg}
			\caption{Arducam社 B0410}
			\label{b0410}
		\end{figure}

			\subsection{デプスカメラの精度}
				これらのデプスカメラの性能をに示す。
				
				\vspace{0.5\baselineskip}
				\begin{table}[h]
					\centering
					\caption{D435の精度}
					\begin{tabular}{|c|r|}
						\hline	動作電圧	& 5V \\
						\hline	消費電流	&	150mA\\
						\hline	視野	&  87°×58°\\
						\hline	測定範囲	&	0.1m~10m\\
						\hline	深度分解能	& 1mm\\
						\hline
					\end{tabular}
					\label{sen5081_spec}
				\end{table}
	\section{デプスカメラの配置と距離の検出方法}
		