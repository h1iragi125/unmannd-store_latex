\graphicspath{{./contents/chapter6/figures/}}	% 図・画像ファイルが保存されたディレクトリまでのパス
\chapter{結論}
本研究では,無人店舗の在庫検出において実現性の高い手法を検討するため,OpenCVを用いた画像処理,超音波センサを用いたセンシング,デプスカメラを用いたセンシングという3つの手法を検討した.さらに,遠隔から在庫状況を把握可能なシステムの構築を目的として,LINE Botを用いた在庫データ共有の実用性を検証した.

OpenCVを用いた画像処理では,白に近い色が特徴的な商品の並ぶ棚に関してはグレースケール変換を,様々な色の商品が並ぶ棚にHSV変換をそれぞれ適用することで,一定の精度の検出が可能となった.しかし,画像上において重なっている商品および密接している商品の検出は困難であった.また,検出には画像内座標を用いるため,実用化の際,画像データ取得用カメラの位置と画角を固定する必要があり,設置が困難という物理的課題が明らかになった.


超音波センサを用いたセンシングでは、商品とセンサとの距離を在庫の個数へと変換し、在庫状況の把握をする手法の確立を行った。これにより、商品の在庫が「大量にある」、「ある」、「ない」の3段階での状況把握が可能となったが、商品の前出しがなされない問題や電力供給の問題が明らかになった。そのため、超音波センサでは在庫把握は困難であるとした。

デプスカメラを用いたセンシングでは、商品の前出しがなされない問題が解決される。だが、カメラを壁面に設置できないと大量のデプスカメラを用意する必要がある。デプスカメラを壁面に設置できれば、カメラの台数を少なくしつつOpenCVを用いた画像処理の手法が使用できる。よって、カメラを壁面に設置したうえでデプスカメラによる在庫把握が最適だと考えたが、依然として在庫の具体的な個数の検出は困難である。

LINE Botを用いた在庫データ共有の実用性の検証では,Raspberry pi3からFlaskサーバへデータを定期的に送信し,格納する手法の確立,加えてMessaging APIやWebhookなどの仕組みを利用し,LINE上で利用者のメッセージに反応して返信する手法の確立を行った.これらを組み合わせることで,無人店舗のLINE公式アカウントから,商品の在庫を利用者の任意のタイミングで確認できる遠隔在庫把握システムの基盤を構築した.しかし,実用化を考えると長時間の安定した稼働が必要であり,Flaskサーバではそれが実現しづらい.また,Messaging APIの通数制限により,実用化によって無人店舗の公式アカウントの利用者が増えるにつれ,必要なコストが増加するという課題が明らかになった.

今後の課題としては,在庫検出手法と遠隔把握手法を組み合わせた遠隔在庫把握システムの構築,運用であり,そのためには本研究で示した検出精度,実現性,コストの大きさといった観点からの課題に取り組み,解消していく必要がある.また,これらの課題に取り組み,遠隔在庫把握システムの運用を実現することで,無人店舗の利便性向上による利用者の増加が期待できる.

