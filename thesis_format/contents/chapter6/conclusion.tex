\graphicspath{{./contents/chapter6/figures/}}	% 図・画像ファイルが保存されたディレクトリまでのパス
\chapter{結論}
本研究では,無人店舗の在庫検出において実現性の高い手法を検討するため,OpenCVを用いた画像処理,超音波センサを用いたセンシング,デプスカメラを用いたセンシングという3つ手法を検討した.さらに,遠隔から在庫状況を把握可能なシステムの構築を目的として,LINE Botを用いた在庫データ共有の実用性を検証した.

OpenCVを用いた画像処理では,白に近い色が特徴的な商品の並ぶ棚に関してはグレースケール変換を,様々な色の商品が並ぶ棚にHSV変換をそれぞれ適用することで,一定の精度の検出が可能となった.しかし,画像上において重なっている商品および密接している商品の検出は困難であった.また,検出には画像内座標を用いるため,実用化の際,画像データ取得用カメラの位置と画角を固定する必要があり,設置が困難という物理的課題が明らかになった.

超音波センサを用いたセンシングでは,

デプスカメラを用いたセンシングでは,
(の理由からデプスカメラによる検出が最適だと考えた.)

LINE Botを用いた在庫データ共有の実用性の検証では,Raspberry pi3からFlaskサーバへデータを定期的に送信し,格納する手法の確立,加えてMessaging APIやWebhookなどの仕組みを利用し,LINE上で利用者のメッセージに反応して返信する手法の確立を行った.これらを組み合わせることで,無人店舗のLINE公式アカウントから,商品の在庫を利用者の任意のタイミングで確認できる遠隔在庫把握システムの基盤を構築した.しかし,実用化を考えると長時間の安定した稼働が必要であり,Flaskサーバではそれが実現しづらい.また,Messaging APIの通数制限により,実用化によって無人店舗の公式アカウントの利用者が増えるにつれ,必要なコストが増加するという課題が明らかになった.

今後の課題としては,在庫検出手法と遠隔把握手法を組み合わせた遠隔在庫把握システムの構築,運用であり,そのためには本研究で示した検出精度,実現性,コストの大きさといった観点からの課題に取り組み,解消していく必要がある.また,これらの課題に取り組み,遠隔在庫把握システムの運用を実現することで,無人店舗の利便性向上による利用者の増加が期待できる.




	% 本章では、(第2章)、第3章で実施した超音波センサによる在庫把握と、第4章で実施したデプスカメラによる在庫把握、(第5章)について考察を行う。
	
	% \section{本研究の限界と今後の課題}
	% 本節では、本研究における限界を明示するとともに、今後の課題について述べる。
	% 	\subsection{本研究の課題}
	% 		\subsubsection{超音波センサに関する課題}
	% 		% 無人店舗で取り扱っている商品は60種類とかなり多い。そのため、超音波センサを1商品に対し1台設置すると多額の費用が発生し、電源供給の課題も発生する。その課題を解決するには、1台のマイコンボードに対し2つ以上の超音波センサを設置するなどの工夫が必要となる。
	% 		% また、無人店舗において、店舗利用者が商品を手に取ったあと、商品棚の奥側にある在庫が前出しされないという問題がある。商品の前出しがされないと実際は在庫数が少ないが在庫数に余裕ありと判断されるため、在庫把握に超音波センサを用いる場合はこの問題を対処する必要がある。この問題を解決するために一部の小売店では\figref{facingstand}のようなフェイシングスタンドを用いているが、無人店舗は商品の種類が多いため導入のハードルが高く、おにぎりのように潰れてしまいやすい商品があるため、導入は厳しい。

	% 		% \vspace{0.5\baselineskip}
	% 		% 	\begin{figure}[h]
	% 		% 		\centering
	% 		% 		\includegraphics[scale=0.5]{sec7maedashi.jpg}
	% 		% 		\caption{フェイシングスタンド}
	% 		% 		\label{facingstand}
	% 		% 	\end{figure}

	% 		\subsubsection{デプスカメラに関する課題}