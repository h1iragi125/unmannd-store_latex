\graphicspath{{./contents/chapter6/figures/}}	% 図・画像ファイルが保存されたディレクトリまでのパス
\chapter{総合的考察}
	本章では、第3章で実施した超音波センサによる在庫把握と、第4章で実施したデプスカメラによる在庫把握、、について考察を行う。
	\section{本研究の限界と今後の課題}
	本節では、本研究における限界を明示するとともに、今後の課題について述べる。
		\subsection{本研究の課題}
			\subsubsection{超音波センサに関する課題}
			無人店舗において、店舗利用者が商品を手に取ったあと、商品棚の奥側にある在庫が前出しされないという問題がある。商品の前出しがされないと在庫数が少ないが在庫数に余裕ありと判断されるため、在庫把握に超音波センサを用いる場合はこの問題を対処する必要がある。この問題を解決するために一部の小売店では\ref{facingstand}のようなフェイシングスタンドを用いているが、無人店舗においては商品の種類が60種類と多いため導入のハードルが高く、おにぎりのように潰れてしまいやすい商品があるため、導入は厳しい。

			\vspace{0.5\baselineskip}
				\begin{figure}[h]
					\centering
					\includegraphics{sec7maedashi.jpg}
					\caption{フェイシングスタンド}
					\label{facingstand}
				\end{figure}

			在庫把握には、装置を大量に用意