\begin{thebibliography}{99} % 二桁分の番号幅を確保
	\setlength{\itemsep}{6pt}	% 文献毎に行間を少し開けるコマンド(不要であればコメントアウト)
		
	% 参考文献定形
	% \bibitem{} ,「」,,最終更新

	% chapter1
	\bibitem{1.1} SpaceCore,「IoTとは?わかりやすく簡単に解説。仕組みや活用法、課題も紹介」,https://space-core.jp/media/14453/,最終更新2024.09.17
	\bibitem{1.2} Degital Intelligence チャンネル,「センシングとは? 基礎から注意点を分かりやすく解説し活用例を紹介」,https://www.cloud-for-all.com/blog/what-is-sensing,最終更新2024.12.11
	\bibitem{1.3} AI総合研究所,「AIとは何か?仕組み・種類・技術用語まで完全ガイド【2025年版】」,https://www.ai-souken.com/article/ai-other,最終更新2025.04.16
	\bibitem{1.4} Rakuten Mobile,「IoTとは?読み方・意味・仕組みや活用事例を簡単に紹介」,https://network.mobile.rakuten.co.jp/sumakatsu/contents/articles/2025/00458/?msockid=1677875e543f61522301935c554a60a2,最終更新2025.12.10
	\bibitem{1.5} TTG,「スマートストアとは?仕組みやメリット、国内の事例を徹底解説」,https://ttg.co.jp/media/what-is-smart-store/,最終更新2025.06.10

	% chapter2
	\bibitem{} ,「」,,最終更新

	% chapter3

	% chapter4

	% chapter5

	% chapter6

	% chapter7

\end{thebibliography}