\graphicspath{{./contents/chapter5/figures/}}	% 図・画像ファイルが保存されたディレクトリまでのパス
\chapter{LINE Botを用いた遠隔の在庫把握}
	第4章で述べたように,無人店舗の在庫状況に関するデータの収集には,デプスカメラを用いる手法が最適と判断した.本章では,収集したデータとLINE Botを用いて,利用者が場所の制約なく無人店舗の在庫状況を閲覧できるシステムの構築について,LINE Bot,ngrok,システムの動作結果と考察を説明する.

    \section{LINE Bot}
    LINE Botは,LINE上で動作する自動応答プログラムであり,利用者から送られる特定の文字や数字に対し,あらかじめ設定したメッセージを自動で返信する機能を持つ\cite{5.1}.また,LINEはLINEヤフー株式会社が提供する日本最大級のコミュニケーションアプリであり,総務省による全年代を対象としたソーシャルメディアの利用率調査では,91.1%の人々がLINEを利用していると回答しているため\cite{5.2},無人店舗の利用者に限らず多くの人が使用経験のあるアプリケーションだと考える.実際に無人店舗の在庫状況を遠隔から確認する際,慣れ親しんだ画面で操作でき,手軽かつ分かりやすく在庫状況を把握できると考えため,本研究ではLINE Botを用いる.

    LINE Botは,主に公式アカウント,Messaging API,Webhookという3つの要素で構成されている.

        \subsection{公式アカウント}
        LINEは,コミュニケーションをとりたいアカウントに対し,友達追加を行うことで,文字や画像,スタンプのやり取りができる.そのため,LINE Botを利用するにあたり,紐づけるアカウントが必要である.これを公式アカウントと呼び,本研究では,無人店舗の在庫状況を返答する公式アカウントを作成し,利用者が友達追加することで,遠隔の在庫把握を実現する.LINE上のやり取りの様子を\figref{yaritori}に示す.

        \vspace{0.5\baselineskip}
		\begin{figure}[H]
			\centering
			\includegraphics[width=.7\linewidth]{図5.1.png}
			\caption{LINE上のやり取りの様子}
			\label{yaritori}
		\end{figure}

        \subsection{Messaging API}
        Messaging APIは,LINE公式アカウントを通じて,利用者と双方向のコミュニケーションを実現するAPIであり,利用者から送られるメッセージに応答する機能を備えている\cite{5.3}.

        API(Application Programming Interface)は,ソフトウェアやWebサービスの間をつなぐインターフェースであり,これを利用することで,外部サービスからのデータ取得を実現できる\cite{5.4}.

        Messaging APIを利用するにあたり,無料で利用できるコミュニケーションプラン,月額5000円のライトプラン,月額15000円のスタンダードプランという,それぞれ返信できるメッセージ数が異なる3つのプランが存在するが,本研究ではLINE Botの有効性を調査するのが目的であり,大量のメッセージを返信する実用段階には満たないため,コミュニケーションプランを用いている.

        \figref{system}に示すLINE Botを用いて在庫状況を把握するシステムの構造から分かる通り,Messaging APIは,Flaskサーバに格納された在庫データを,LINEプラットフォームを通じて利用者に送信するために用いる.ここで,LINEプラットフォームは,LINEヤフー株式会社が提供する通信および制御を行うための基盤であり,Messaging APIはこの基盤上で提供されるサービスの1つである.また, Flaskサーバは,Pythonプログラミング言語を用いたWebアプリケーションフレームワークであり,シンプルかつ軽量,さらに柔軟性や拡張性が高いという特徴がある\cite{5.5}.本研究では,LINE Botの有効性を調査するのが目的のため,検出した在庫データを保持するサーバとして十分な機能を持ち,無料で利用できるFlaskサーバを用いた.

        \vspace{0.5\baselineskip}
		\begin{figure}[H]
			\centering
			\includegraphics[width=.8\linewidth]{図5.2.png}
			\caption{遠隔在庫把握システムの構造}
			\label{system}
		\end{figure}

        Messaging APIを利用するためには,\figref{access}に示すLINE BotのMessaging API設定画面(以下LINE Developers画面とする)のように,チャネルアクセストークンの発行が必要である.これは,Messaging APIを安全に利用するための認証キーであり,本研究ではFlaskサーバからLINEプラットフォームへメッセージ送信を行うための身分証としての役割を果たす.\figref{access}は,セキュリティ上の理由から,チャネルアクセストークンを隠して示している.

        \vspace{0.5\baselineskip}
		\begin{figure}[H]
			\centering
			\includegraphics[width=.9\linewidth]{図5.3.png}
			\caption{チャネルアクセストークン発行画面}
			\label{access}
		\end{figure}

        \subsection{Webhookとngrok}
        Webhookは,特定のイベントが発生した際,あらかじめ設定したアプリケーションに通知を自動で送信する仕組みであり\cite{5.6},\figref{system}に示した遠隔在庫把握システムの構造から分かる通り,本研究では,在庫データを管理するFlaskサーバに対し,イベントの発生を通知する際に用いる.遠隔在庫把握システムにおけるイベントとは,利用者からのメッセージを指し,Webhookを利用することで,利用者のメッセージに対して応答する仕組みを実現できる.

        Webhookを利用し,イベントの発生を通知するにあたり,ローカル環境で動作するサーバから外部ネットワークへの通信はツールを用いず実現できるが,外部ネットワークからローカル環境への通信は特定のツールが必要という問題が存在する.この問題を解決するツールの1つにngrokがあり\cite{5.7},\figref{system}に示した遠隔在庫把握システムの構造から分かる通り,本研究ではインターネット上のLINEプラットフォームから学内LAN内にあるFlaskサーバにWebhookを送信するためngrokを用いる.このツールの特徴として,無料かつ簡単に利用できる点が挙げられる.また,ngrokは,より便利な機能を備えた有償版もあるが,本研究では,LINE Botの有効性調査に支障のない無料版を用いる.

        LINE Botにおけるngrokを用いたWebhookの具体的な設定方法について説明する.テキストベースのプログラム実行ツール(以下ターミナルとする)において,ソースコード\ref{source5.1}に示すFlaskサーバのポート公開コードのように,ポート番号を5000としてngrokに公開URLを生成させる.ここで,ポート番号を5000としているのは,Flaskサーバのデフォルト設定がこの番号のためであり\cite{5.8},任意に変更することができるが,本研究ではデフォルトの番号を用いている.生成される公開URLを\figref{URL}に示すが,セキュリティ上の理由から,URLを隠して示している.

        \begin{lstlisting}[language=Python, caption=Flaskサーバのポート公開用コード, label=source5.1]
ngrok http 5000
		\end{lstlisting}

        \vspace{0.5\baselineskip}
		\begin{figure}[H]
			\centering
			\includegraphics[width=.9\linewidth]{図5.4.png}
			\caption{ngrokで生成される公開URL}
			\label{URL}
		\end{figure}

        \figref{URL}に示したngrokによって生成される公開URLに「/callback」を付与したものを,\figref{Webhook}に示すLINE DevelopersのWebhook設定画面のように,Webhook URLとして設定し,Webhookの利用をONにする.ここで,「/ callback」は,本研究においてFlaskサーバがWebhookを受け取るための窓口を示しており,利用者のメッセージに沿った応答をするために必要なコードである.また,Webhook利用時に自動で発行されるチャネルシークレットという文字列を用いてWebhookの認証を行う.チャネルシークレットはLINE Developersから確認できる.これらの作業によって,公式アカウントに利用者がメッセージを送信した際,インターネット上のLINEプラットフォームを通じて,学内LAN内のFlaskサーバへイベントの発生を通知できる.「/callback」を用いたWebhookの窓口設定に関するコードをソースコード\ref{source5.2}に示す.このコードにおいて,signatureはWebhookの通知がLINEプラットフォームから送られた正規のものであるか,改ざんされていないかを認証するための署名であり,bodyはメッセージ内容,ユーザIDなどのイベント情報を文字列で取得するためのコードである.また,\figref{Webhook}はセキュリティ上の理由から,Webhook URLを隠して示している.

        \vspace{0.5\baselineskip}
		\begin{figure}[H]
			\centering
			\includegraphics[width=.9\linewidth]{図5.5.png}
			\caption{Webhook設定画面}
			\label{Webhook}
		\end{figure}

        \begin{lstlisting}[language=Python, caption=Webhookの窓口設定用コード, label=source5.2]
from flask import Flask, request, abort
from linebot import WebhookHandler

# Webhook受信
@app.route("/callback", methods=['POST'])
def callback():

    # LINEからの改ざん防止用署名を取得
    signature = request.headers['X-Line-Signature']

    # イベント情報(JSON形式)を取得
    body = request.get_data(as_text=True)
    print("Received body:", body)

    # 処理の受け渡し
    try:
        handler.handle(body, signature)

    # エラー発生時の処理
    except Exception as e:
        print("[ERROR] LINE handler:", e)
        abort(400)

    return 'OK'
		\end{lstlisting}
    
    \section{遠隔在庫把握システムの流れ}
    まず,Flaskサーバに在庫データを格納する仕組みについて説明する.\figref{system}に示した遠隔在庫把握システムの構造から分かる通り,デプスカメラで検出した商品在庫の個数情報を,Raspberry pi3を用いてJSON形式でFlaskサーバに送信する.この際,Flaskサーバに保持するデータは各商品1つであり,10秒間隔で最新のデータに更新する.ここで,最新のデータのみを格納している理由としては,Flaskサーバの処理を軽くするためである.また,素早く返信を行うために,利用者からのメッセージを検知してFlaskサーバがデータ取得を始めるのではなく,Flaskサーバにあらかじめ格納されている最新の在庫データを用いて返信を行う.Raspberry pi3に実装するFlaskサーバへデータを送信するためのコードをソースコード\ref{source5.3}に示し,Flaskサーバ側に実装するRaspberry pi3から送信されるデータ受信用のコードをソースコード\ref{source5.4}に示す.本研究では,LINE Botの有効性調査のため,実際に検出された値ではなく,1つの商品の検出値に見立てた実験用のダミー値を利用している.実用化の際には、無人店舗の全商品の在庫を検出,データとして格納することを想定している.注意点として,Flaskサーバを立てるPCとRaspberry pi3は同一LAN内で通信を行う必要がある.本研究では,学内LANを用いている.また,セキュリティ上の理由から,ソースコード\ref{source5.3}においてFlaskサーバを立てたPCのIPアドレスは,実際のものではなく,例を示している.

    \begin{lstlisting}[language=Python, caption=Raspberry pi3におけるFlaskサーバへのデータ送信用コード, label=source5.3]
import requests
import json
import time

# サーバのURL指定
FLASK_SERVER_URL = "http:// {サーバを立てたPCのIPアドレス(IPv4)}:5000/upload"

# データ送信準備
def send_data_to_server(value):
    data = {"value": value}
    try:
        res = requests.post(FLASK_SERVER_URL, json=data)
        print(f"[SEND] Sent{value}, Status: {res.status_code}")

    # エラー処理
    except Exception as e:
        print(f"[ERROR] {e}")

# データ定期送信
if __name__ == "__main__":
    while True:

        # センサ値のダミーデータ
        fake_value = 5
        send_data_to_server(fake_value)
        time.sleep(10)
    \end{lstlisting}

    \begin{lstlisting}[language=Python, caption=Flaskサーバ側に実装するRaspberryからのデータ受信用コード, label=source5.4]
from flask import Flask, request, jsonify
from datetime import datetime

# Raspberry pi3からデータ受信
@app.route("/upload", methods=['POST'])
def upload():

    # JSON形式でデータ受信
    try:
        data = request.get_json()

	# データの妥当性確認
        if not data or "value" not in data:
            return jsonify({"error": "Invalid JSON"}), 400

	# データ取得
        value = data["value"]

	# 最新データに更新
        latest_data["timestamp"] = datetime.now().strftime("%Y-%m-%d %H:%M:%S")
        latest_data["value"] = value

        print(f"[UPLOAD] Received new data from device: {value}")

        return jsonify({"status": "success", "received_value": value}), 200
    except Exception as e:
        print("[ERROR] Upload failed:", e)
        return jsonify({"error": str(e)}), 500
    \end{lstlisting}

    ソースコード\ref{source5.5}に示すWebブラウザを用いたデータ閲覧のためのコードを用いて,Flaskサーバに格納される最新の在庫データを可視化する.\figref{kasika}示す可視化したデータから分かる通り,商品名,更新時間,数量を可視化している.

    \begin{lstlisting}[language=Python, caption=ブラウザ上でのデータ閲覧用コード, label=source5.5]
from flask import Flask

@app.route("/status")

# HTMLで記述
def status():
    html = (
    <!DOCTYPE html>
    <html lang="ja">
    <head>
        <meta charset="UTF-8">
        <title>最新データ</title>
        <style>
            body {{
                font-family: "Segoe UI", sans-serif;
                background-color: #ffffff;
                color: #000000;
                padding: 50px;
            }}
            h1 {{
                font-size: 28px;
                font-weight: 700; /* 太め */
                margin-bottom: 20px;
            }}
            ul {{
                list-style-type: none;
                padding-left: 10px;
                font-size: 18px;
                font-weight: 400; /* 通常の太さ */
                line-height: 1.8;
            }}
        </style>
    </head>
    <body>
        <h1>最新データ</h1>
        <ul>
            <li>商品名:ブルーベリー&クリームチーズベーグル</li>
            <li>時間:{latest_data['timestamp']}</li>
            <li>数量:{latest_data['value']}</li>
        </ul>
    </body>
    </html>
    )
    return html
    \end{lstlisting}

    \vspace{0.5\baselineskip}
	\begin{figure}[H]
		\centering
		\includegraphics[width=.9\linewidth]{図5.6.png}
		\caption{ブラウザで可視化した在庫データ}
		\label{kasika}
	\end{figure}

    Flaskサーバに最新の在庫データが格納されることを前提に,LINE公式アカウントに利用者がメッセージを送信してから返信されるまでの,LINEプラットフォームとFLASKサーバ間のMessaging APIとWebhookの通信の流れについて説明する.\figref{system}に示した遠隔在庫把握システムの構造から分かる通り,利用者がメッセージを送信した際,そのメッセージはLINEプラットフォームが受け取る.その後,6.1.3節で述べたように,ngrokによる Webhookの利用で,Flaskサーバへイベント発生の通知を行い,利用者からのメッセージを認識,その内容があらかじめ定めたテキストと一致した場合,自動で返信を行う.この返信には6.1.2節で述べたように,Messaging APIが用いられ,本研究では商品名を「ブルーベリー&クリームチーズベーグル」とし,「「ブルーベリー&クリームチーズベーグル」の在庫」というメッセージを受信した際にその個数を返答するよう設定,検証を行う.この返答は,LINEプラットフォームを通じて,利用者へ送信される.利用者からのメッセージテキストを認識し,Messaging APIを用いて返答するためのコードをソースコード\ref{source5.6}に示す.ここで,LINE公式アカウントの認証に用いるチャネルアクセストークンとチャネルシークレットは,セキュリティ上の理由から伏せる.

    \begin{lstlisting}[language=Python, caption=LINEプラットフォームを通して利用者へ返信するためのコード, label=source5.6]
from linebot import LineBotApi, WebhookHandler
from linebot.models import MessageEvent, TextMessage, TextSendMessage

# 認証情報の設定
LINE_CHANNEL_ACCESS_TOKEN = '' 
LINE_CHANNEL_SECRET = '' 
line_bot_api = LineBotApi(LINE_CHANNEL_ACCESS_TOKEN)
handler = WebhookHandler(LINE_CHANNEL_SECRET)

# 最新データを保持する変数
latest_data = {"ブルーベリー&クリームチーズベーグル":{"timestamp": None, "value": None}}

@handler.add(MessageEvent, message=TextMessage)
def handle_message(event):
    text = event.message.text
    if text == "「ブルーベリー&クリームチーズベーグル」の在庫":
        reply = f"現在の「ブルーベリー&クリームチーズベーグル」は、{latest_data['value']}個です。"
        line_bot_api.reply_message(event.reply_token, TextSendMessage(text=reply))
    \end{lstlisting}

    \section{公式アカウント上のメッセージ送信手法}
    本研究では,LINE公式アカウント上で利用者が在庫状況を確認する際,商品の選択を行うためのメッセージ送信手法として,リッチメニューおよびリッチメッセージを用いる.これは,画像およびテキストによって表現されたバナーを利用者が選択することにより,バナーごとに設定された固定メッセージが自動で送信されるものである.リッチメニューおよびリッチメッセージの特徴として,視覚的に情報を表現できること,テキストを入力する必要がなく,公式アカウント管理者が設定したメッセージを利用者に送信させられることの2つが挙げられる.また,リッチメニューは\figref{menu}に示すように,トーク画面を開いたときに表示される選択式バナーであり,リッチメッセージは\figref{message}に示すように,リッチメニューを選択したとき,公式アカウントの返信として表示される選択式バナーである.

    \vspace{0.5\baselineskip}
	\begin{figure}[H]
		\centering
		\begin{minipage}{.45\linewidth}
			\centering
			\includegraphics[width=\linewidth]{図5.7.png}
			\caption{リッチメニュー表示}
			\label{menu}
		\end{minipage}
		\hfill
		\vspace{0.5\baselineskip}
		\begin{minipage}{.45\linewidth}
			\centering
			\includegraphics[width=\linewidth]{図5.8.png}
			\caption{リッチメッセージ表示}
			\label{message}
		\end{minipage}
	\end{figure}

    無人店舗の商品は数が多いため,商品名とその見た目を記憶するのは困難である.そのため,\figref{menu}および\figref{message}に示す実際のLINE公式アカウントのトーク画面から分かる通り,リッチメニューおよびリッチメッセージを用いて商品棚や商品名の見た目と名称を同時に表示することで,利用者の想定する商品を選択することができる.

    6.2節で述べたように,在庫データを返信するには,利用者の送信するメッセージテキストが,あらかじめ定めたテキストに完全に一致している必要がある.\figref{menutap}および\figref{messagetap}に示すLINE公式アカウント上のやり取りから分かる通り,リッチメニューおよびリッチメッセージはバナーをタップすると必ず固定のメッセージが送信され,そのメッセージは公式アカウント管理者が設定できる.そのため,これらを利用することで,迅速かつ正確な在庫データの提供が可能になる.

    \vspace{0.5\baselineskip}
	\begin{figure}[H]
		\centering
		\begin{minipage}{.45\linewidth}
			\centering
			\includegraphics[width=\linewidth]{図5.9.png}
			\caption{リッチメニューのタップ操作}
			\label{menutap}
		\end{minipage}
		\hfill
		\vspace{0.5\baselineskip}
		\begin{minipage}{.45\linewidth}
			\centering
			\includegraphics[width=\linewidth]{図5.10.png}
			\caption{リッチメッセージのタップ操作}
			\label{messagetap}
		\end{minipage}
	\end{figure}

    リッチメニューおよびリッチメッセージの設定方法について説明する.\figref{menusettei}に示すLINE公式アカウントのリッチメニュー設定画面から,リッチメニューの表現テンプレート,適用する画像およびテキスト,タップ操作時のアクション,メニューバーをそれぞれ設定する.これに表示期間を設定し,公開することで \figref{menu}に示したように,トーク画面を開いた際にリッチメニューが表示される.

    \vspace{0.5\baselineskip}
	\begin{figure}[H]
		\centering
		\includegraphics[width=.9\linewidth]{図5.11.png}
		\caption{リッチメニュー設定画面}
		\label{menusettei}
	\end{figure}

    \figref{messagesettei}に示すLINE公式アカウントのリッチメッセージ設定画面から,リッチメッセージの表現テンプレート,適用する画像およびテキスト,タップ操作時のアクションをそれぞれ設定する.その後,\figref{return}に示すLINE公式アカウントの応答メッセージ設定画面から,リッチメッセージが返信として表示されるためのキーワードを設定することで,図6.8に示したように,利用者がキーワードを公式アカウントに送信した際,対応するリッチメッセージが表示される.利用者のキーワード送信を,リッチメニューおよび別のリッチメッセージでタップ操作した際のアクションに設定することで,タップ操作のみで特定の商品に辿り着くことができる.リッチメニューおよびリッチメッセージを利用した商品検索の流れを\figref{search}に示す.

    \vspace{0.5\baselineskip}
	\begin{figure}[H]
		\centering
		\includegraphics[width=.9\linewidth]{図5.12.png}
		\caption{リッチメッセージ設定画面}
		\label{messagesettei}
	\end{figure}

    \vspace{0.5\baselineskip}
	\begin{figure}[H]
		\centering
		\includegraphics[width=.9\linewidth]{図5.13.png}
		\caption{応答メッセージ設定画面}
		\label{return}
	\end{figure}

    \vspace{0.5\baselineskip}
	\begin{figure}[H]
		\centering
		\includegraphics[width=.95\linewidth]{図5.14.png}
		\caption{商品検索の流れ}
		\label{search}
	\end{figure}

    \section{システムの動作結果と考察,課題}
    \figref{kasika}に示したFlaskサーバの格納データ可視化画面から分かる通り,Raspberry pi3からFlaskサーバへのデータ送信および格納については,正確に行うことができた.

    6.2節で述べたように,Messaging API,Webhook,ngrokを用いて,利用者が「ブルーベリー&クリームチーズベーグル」の在庫を把握するため,LINE公式アカウントにメッセージを送信した際の結果を\figref{result}に示す.この結果から,LINE Botを用いた在庫データの把握が可能であると判断した.

    \vspace{0.5\baselineskip}
	\begin{figure}[H]
		\centering
		\includegraphics[width=.5\linewidth]{図5.15.png}
		\caption{遠隔在庫把握の結果}
		\label{result}
	\end{figure}

    6.2節で述べたように,本研究では,処理の軽量化のためFlaskサーバに最新のデータのみを格納している.しかし,過去の在庫データの履歴を時刻情報などと併せて保持することで,遠隔の在庫把握以外,例えば,商品の時刻ごとの売れ行きや天気,季節による売れ行きの変化など,様々な目的に応用できると考える.

    本研究では,利用者がメッセージを送信したときに在庫を返信するシステムに関して,LINE Botによる実現性を確認するため在庫データの格納および中継方法にFlaskサーバを用いた.しかし,実用化の際は,常時,もしくは長時間の運用が考えられるため,本研究と同様にPC上のFlaskサーバを用いると,1台のPCをRaspberry pi3と同じLAN内で起動させ続ける必要がある.そこで,Raspberry pi上でFlaskサーバを起動する方法が考えられる.これは,検出を行うRaspberry pi3自体を用いても可能だが,負荷分散のため検出用とは別のRaspberry piが適切であると考える.また,より安定的な稼働のため,AWS,Microsoft Azure,Google Cloudといったクラウドサービスの利用も考えられる.

    このシステムの実用化にあたり,2つの課題がある.

    1つ目は,Messaging APIの使用に関する通数制限である.6.1.2節で述べたようにMessaging APIには,3つのプランがあり,それぞれで送信できるメッセージの数量が変わる.具体的には,無料のコミュニケーションプランで月200通,月5000円のライトプランで月5000通,月15000円のスタンダードプランで月30000通となっており、この通数を超過すると,コミュニケーションプランおよびスタンダードプランではLINE公式アカウントからメッセージを送信ができなくなり,スタンダードプランでは利用できるが,1通の超過に対し3円の追加料金が発生する\cite{5.9}.そのため,無人店舗の遠隔在庫把握システムを多くの人が利用する場合,LINE Botを用いると費用が大きくなると考えられる.ここで使用している通数という単位は$通数 = 送信メッセージ数 \times 送信先アカウント数$で計算できる.
    
    2つ目は,PCでLINEを利用する場合,リッチメニューが表示されない点である.図5.16に示すPC版LINEで表示した公式アカウントのトーク画面から分かる通り,リッチメニューはモバイル専用機能であり,PC版およびWeb版のLINEでは表示できない\cite{5.10}.そのため,実用化の際,モバイル機器以外を用いる利用者に向けて遠隔在庫把握システムを構築する場合は,本章で説明したリッチメニューを利用した商品検索とは別の手法を用いる必要がある.
    
