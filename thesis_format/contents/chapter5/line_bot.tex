\graphicspath{{./contents/chapter5/figures/}}	% 図・画像ファイルが保存されたディレクトリまでのパス
\chapter{LINE Botを用いた遠隔の在庫把握}
	第4章で述べたように,無人店舗の在庫状況に関するデータの収集には,デプスカメラを用いる手法が最適と判断した.本章では,収集したデータとLINE Botを用いて,利用者が場所の制約なく無人店舗の在庫状況を閲覧できるシステムの構築について,LINE Bot,ngrok,システムの動作結果と考察を説明する.

    \section{LINE Bot}
    LINE Botは,LINE上で動作する自動応答プログラムであり,利用者から送られる特定の文字や数字に対し,あらかじめ設定したメッセージを自動で返信する機能を持つ[16].また,LINEはLINEヤフー株式会社が提供する日本最大級のコミュニケーションアプリであり,総務省による全年代を対象としたソーシャルメディアの利用率調査では,91.1%の人々がLINEを利用していると回答しているため[17],無人店舗の利用者に限らず多くの人が使用経験のあるアプリケーションだと考える.実際に無人店舗の在庫状況を遠隔から確認する際,慣れ親しんだ画面で操作でき,手軽かつ分かりやすく在庫状況を把握できると考えため,本研究ではLINE Botを用いる.
    LINE Botは,主に公式アカウント,Messaging API,Webhookという3つの要素で構成されている.

        \subsection{公式アカウント}
        LINEは,コミュニケーションをとりたいアカウントに対し,友達追加を行うことで,文字や画像,スタンプのやり取りができる.そのため,LINE Botを利用するにあたり,紐づけるアカウントが必要である.これを公式アカウントと呼び,本研究では,無人店舗の在庫状況を返答する公式アカウントを作成し,利用者が友達追加することで,遠隔の在庫把握を実現する.LINE上のやり取りの様子を\figref{yaritori}に示す.

        \vspace{0.5\baselineskip}
		\begin{figure}[H]
			\centering
			\includegraphics[width=.7\linewidth]{図5.1.png}
			\caption{LINE上のやり取りの様子}
			\label{yaritori}
		\end{figure}

        \subsection{Messaging API}
        Messaging APIは,LINE公式アカウントを通じて,利用者と双方向のコミュニケーションを実現するAPIであり,利用者から送られるメッセージに応答する機能を備えている[18].
        API(Application Programming Interface)は,ソフトウェアやWebサービスの間をつなぐインターフェースであり,これを利用することで,外部サービスからのデータ取得を実現できる[19].
        Messaging APIを利用するにあたり,無料で利用できるコミュニケーションプラン,月額5000円のライトプラン,月額15000円のスタンダードプランという,それぞれ返信できるメッセージ数が異なる3つのプランが存在するが,本研究ではLINE Botの有効性を調査するのが目的であり,大量のメッセージを返信する実用段階には満たないため,コミュニケーションプランを用いている.
        \figref{system}に示すLINE Botを用いて在庫状況を把握するシステムの構造から分かる通り,Messaging APIは,Flaskサーバに格納された在庫データを,LINEプラットフォームを通じて利用者に送信するために用いる.ここで,LINEプラットフォームは,LINEヤフー株式会社が提供する通信および制御を行うための基盤であり,Messaging APIはこの基盤上で提供されるサービスの1つである.また, Flaskサーバは,Pythonプログラミング言語を用いたWebアプリケーションフレームワークであり,シンプルかつ軽量,さらに柔軟性や拡張性が高いという特徴がある[20].本研究では,LINE Botの有効性を調査するのが目的のため,検出した在庫データを保持するサーバとして十分な機能を持ち,無料で利用できるFlaskサーバを用いた.

        \vspace{0.5\baselineskip}
		\begin{figure}[H]
			\centering
			\includegraphics[width=.8\linewidth]{図5.2.png}
			\caption{遠隔在庫把握システムの構造}
			\label{system}
		\end{figure}

        Messaging APIを利用するためには,\figref{access}に示すLINE BotのMessaging API設定画面(以下LINE Developers画面とする)のように,チャネルアクセストークンの発行が必要である.これは,Messaging APIを安全に利用するための認証キーであり,本研究ではFlaskサーバからLINEプラットフォームへメッセージ送信を行うための身分証としての役割を果たす.\figref{access}は,セキュリティ上の理由から,チャネルアクセストークンを隠して示している.

        \vspace{0.5\baselineskip}
		\begin{figure}[H]
			\centering
			\includegraphics[width=.8\linewidth]{図5.3.png}
			\caption{チャネルアクセストークン発行画面}
			\label{access}
		\end{figure}

        \subsection{Webhookとngrok}
        Webhookは,特定のイベントが発生した際,あらかじめ設定したアプリケーションに通知を自動で送信する仕組みであり[21],\figref{system}に示した遠隔在庫把握システムの構造から分かる通り,本研究では,在庫データを管理するFlaskサーバに対し,イベントの発生を通知する際に用いる.遠隔在庫把握システムにおけるイベントとは,利用者からのメッセージを指し,Webhookを利用することで,利用者のメッセージに対して応答する仕組みを実現できる.
        Webhookを利用し,イベントの発生を通知するにあたり,ローカル環境で動作するサーバから外部ネットワークへの通信はツールを用いず実現できるが,外部ネットワークからローカル環境への通信は特定のツールが必要という問題が存在する.この問題を解決するツールの1つにngrokがあり[22],\figref{system}に示した遠隔在庫把握システムの構造から分かる通り,本研究ではインターネット上のLINEプラットフォームから学内LAN内にあるFlaskサーバにWebhookを送信するためngrokを用いる.このツールの特徴として,無料かつ簡単に利用できる点が挙げられる.また,ngrokは,より便利な機能を備えた有償版もあるが,本研究では,LINE Botの有効性調査に支障のない無料版を用いる.
        LINE Botにおけるngrokを用いたWebhookの具体的な設定方法について説明する.テキストベースのプログラム実行ツール(以下ターミナルとする)において,\ref{source5.1}に示すFlaskサーバのポート公開コードのように,ポート番号を5000としてngrokに公開URLを生成させる.ここで,ポート番号を5000としているのは,Flaskサーバのデフォルト設定がこの番号のためであり[23],任意に変更することができるが,本研究ではデフォルトの番号を用いている.生成される公開URLを\figref{URL}に示すが,セキュリティ上の理由から,URLを隠して示している.

        \begin{lstlisting}[language=Python, caption=ソースコード5.1, label=source5.1]
ngrok http 5000
		\end{lstlisting}

        \vspace{0.5\baselineskip}
		\begin{figure}[H]
			\centering
			\includegraphics[width=.9\linewidth]{図5.4.png}
			\caption{ngrokで生成される公開URL}
			\label{URL}
		\end{figure}

        \figref{URL}に示したngrokによって生成される公開URLに「/callback」を付与したものを,\figref{Webhook}に示すLINE DevelopersのWebhook設定画面のように,Webhook URLとして設定し,Webhookの利用をONにする.ここで,「/ callback」は,本研究においてFlaskサーバがWebhookを受け取るための窓口を示しており,利用者のメッセージに沿った応答をするために必要なコードである.これらの作業によって,公式アカウントに利用者がメッセージを送信した際,インターネット上のLINEプラットフォームを通じて,学内LAN内のFlaskサーバへイベントの発生を通知できる.「/callback」を用いたWebhookの窓口設定に関するコードを\ref{source5.2}に示す.このコードにおいて,signatureはWebhookの通知がLINEプラットフォームから送られた正規のものであるか,改ざんされていないかを認証するための署名であり,bodyはメッセージ内容,ユーザIDなどのイベント情報を文字列で取得するためのコードである.また,\figref{Webhook}はセキュリティ上の理由から,Webhook URLを隠して示している.

        \vspace{0.5\baselineskip}
		\begin{figure}[H]
			\centering
			\includegraphics[width=.8\linewidth]{図5.5.png}
			\caption{Webhook設定画面}
			\label{Webhook}
		\end{figure}

        \begin{lstlisting}[language=Python, caption=ソースコード5.2, label=source5.2]
from flask import Flask, request, abort
from linebot import WebhookHandler

# Webhook受信
@app.route("/callback", methods=['POST'])
def callback():

    # LINEからの改ざん防止用署名を取得
    signature = request.headers['X-Line-Signature']

    # イベント情報(JSON形式)を取得
    body = request.get_data(as_text=True)
    print("Received body:", body)

    # 処理の受け渡し
    try:
        handler.handle(body, signature)

    # エラー発生時の処理
    except Exception as e:
        print("[ERROR] LINE handler:", e)
        abort(400)

    return 'OK'
		\end{lstlisting}
    
    \section{遠隔在庫把握システムの流れ}
    まず,Flaskサーバに在庫データを格納する仕組みについて説明する.\figref{system}に示した遠隔在庫把握システムの構造から分かる通り,デプスカメラで検出した商品在庫の個数情報を,Raspberry pi3を用いてJSON形式でFlaskサーバに送信する.この際,Flaskサーバに保持するデータは各商品1つであり,10秒間隔で最新のデータに更新する.ここで,最新のデータのみを格納している理由としては,Flaskサーバの処理を軽くするためである.また,素早く返信を行うために,利用者からのメッセージを検知してFlaskサーバがデータ取得を始めるのではなく,Flaskサーバにあらかじめ格納されている最新の在庫データを用いて返信を行う.Raspberry pi3に実装するFlaskサーバへデータを送信するためのコードをソースコード6.2に示す.本研究では,LINE Botの有効性調査のため,実際に検出された値ではなく,1つの商品の検出値に見立てた実験用のダミー値を利用している.実用化の際には、無人店舗の全商品の在庫を検出,データとして格納することを想定している.注意点として,Flaskサーバを立てるPCとRaspberry pi3は同一LAN内で通信を行う必要がある.本研究では,学内LANを用いている.また,セキュリティ上の理由から,ソースコード6.2においてFlaskサーバを立てたPCのIPアドレスは,実際のものではなく,例を示している.
