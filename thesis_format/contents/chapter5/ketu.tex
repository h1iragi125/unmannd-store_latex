\chapter{結論}
	サーマルカメラを用いた混雑状況のモニタリング手法を開発するために、輪郭抽出及び機械学習を用いて赤外線画像内の人数を数えるシステムを開発した。モデルケースとして想定する成蹊大学の第一学生食堂での撮影状況を考慮し、研究室内での赤外線画像の撮影を行った。輪郭抽出による赤外線画像内の人数把握により撮影範囲内の混雑状況を推定することを試みたが、ノイズや人以外の熱源等の輪郭を抽出してしまい、正確に人数を把握することができない。従って、別の手法として機械学習を用いることにより、画像の特徴からの人数把握を試みた。機械学習では画像処理に適したCNNを使用し、0~3人を撮影した計130枚の赤外線画像を用いてモデルを作成し、学習を行った。作成したモデルを用いて赤外線画像内の人数予測を行った。
	
	機械学習による人数予測の正解率は94.87\%となった。また、適合率、再現率、F1スコアの評価指数もそれぞれ90\%を超えたため、モデルの正確性が高いと考えられる。
	しかし、今回行った人数把握実験で使用した赤外線画像は研究室内で撮影したものであり、3人以上の場合や人が重なっている場合は考慮されていない。また、食堂への設置は長期間を想定しているが、季節の変化等による温度変化の影響ついても考慮されていない。そのため、実際に食堂へ測定システムを設置し、長期的に撮影を行うことで多人数や人が重なった場合等の多様なデータを用いてモデルの精度や対応力を向上させる必要がある。

	