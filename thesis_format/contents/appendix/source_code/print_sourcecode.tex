% appendixで示すコードののディレクトリパス(絶対パスが使えなくなる & 一度に1つしか指定できないので注意)
\lstset{inputpath=./contents/appendix/source_code/program_files/}
\chapter{プログラム}

	\section{画像処理のソースコード}
	以下はソースコードである。OpenCVを用いた画像処理の在庫検出において,グレースケール変換を利用した検出プログラムを\ref{gray}に示す.
	\hypertarget{measurepy}{% ハイパーリンク先の設定\hyperlink{label}{}でここに飛ぶ(内部のlabelと合わせると便利かも)
			\lstinputlisting[style=mypystyle,caption=グレースケール画像の検出プログラム,label=gray]{grayscale.py}
		}

	OpenCVを用いた画像処理の在庫検出において,HSV変換を利用した検出プログラムを\ref{color}に示す.
	\hypertarget{measurepy}{% ハイパーリンク先の設定\hyperlink{label}{}でここに飛ぶ(内部のlabelと合わせると便利かも)
			\lstinputlisting[style=mypystyle,caption=HSV画像の検出プログラム,label=color]{colorful.py}
		}

	\section{超音波センサのソースコード}
		超音波センサでの在庫数把握において,マイコンボードで実行したプログラムが\ref{measurepy}である.
		\hypertarget{measurepy}{% ハイパーリンク先の設定\hyperlink{label}{}でここに飛ぶ(内部のlabelと合わせると便利かも)
			\lstinputlisting[style=mypystyle,caption=findContours,label=measurepy]{measure.py}
		}
		超音波センサ
		\ref{machine}に機械学習に用いた一連のプログラムを示す.
		\hypertarget{machine}{% ハイパーリンク先の設定\hyperlink{label}{}でここに飛ぶ(内部のlabelと合わせると便利かも)
			\lstinputlisting[style=mypystyle,caption=MachineLearning,label=machine]{countpeople.py}
		}

	\section{デプスカメラのソースコード}


	\section{LINE Botのソースコード}
	LINE Botを用いた在庫データの共有について,Raspberry pi3からFlaskサーバへデータを送信するためのプログラムを\ref{datasend}に示す.
	\hypertarget{machine}{% ハイパーリンク先の設定\hyperlink{label}{}でここに飛ぶ(内部のlabelと合わせると便利かも)
			\lstinputlisting[style=mypystyle,caption=Raspberry pi3とFlaskサーバ間のデータ送信プログラム,label=datasend]{between_pi_flask.py}
		}

	LINE Botを用いた在庫データの共有について,公式アカウント利用者からの在庫に関するメッセージに対し,FlaskサーバからLINEプラットフォームを通して返信を行うプログラムを\ref{message}に示す.
	\hypertarget{machine}{% ハイパーリンク先の設定\hyperlink{label}{}でここに飛ぶ(内部のlabelと合わせると便利かも)
			\lstinputlisting[style=mypystyle,caption=利用者への在庫通知プログラム,label=message]{LINEBot.py}
		}
	