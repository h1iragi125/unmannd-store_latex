\graphicspath{{./contents/chapter2/figures/}}	% 図・画像ファイルが保存されたディレクトリまでのパス
\chapter{画像処理による在庫検出}
無人店舗の在庫状況に関するデータの収集を実現するため,複数の技術的アプローチについて検討を行い,最適な在庫検出手法を明確にする.本章では,画像処理技術を用いた在庫検出について,OpenCVを用いた検出方法,輪郭抽出,座標による商品識別,検出した結果と考察を説明する.
	\section{画像取得に用いた機器}
		画像処理を行うにあたり,無人店舗の商品棚の画像を取得するため,スマートフォンのカメラ機能を利用した.本章では,画像処理技術を用いた在庫検出の実現性について検討することを主な目的としているため,コスト削減や画像取得の容易さの観点からこの選択をした.実際の画像取得に用いたスマートフォンを図\figref{Camera}に,取得した商品棚の画像を図\figref{onigiri},図\figref{colorful}に示す.
		% [width=.5\linewidth]
		\vspace{0.5\baselineskip}
		\begin{figure}[h]
			\centering
			\includegraphics[width=.4\linewidth]{図2.1.png}
			\caption{画像取得用スマートフォン}
			\label{Camera}
		\end{figure}

		\vspace{0.5\baselineskip}
		\begin{figure}[htbp]
			\centering
			\begin{minipage}{.45\linewidth}
				\centering
				\includegraphics[width=\linewidth]{図2.2.jpg}
				\caption{おにぎりを中心とする商品棚}
				\label{onigiri}
			\end{minipage}
			\hfill
			\vspace{0.5\baselineskip}
			\begin{minipage}{.45\linewidth}
				\centering
				\includegraphics[width=\linewidth]{図2.3.png}
				\caption{様々な色の商品が並ぶ棚}
				\label{colorful}
			\end{minipage}
		\end{figure}

			\subsection{サーマルカメラの精度}
				MLX90640の性能を\tabref{MLX90640}に示す。測定範囲は-40℃~+300℃であり、混雑状況のモニタリングの際に観測される温度は測定可能であると考えられる。
				
				\vspace{0.5\baselineskip}
				\begin{table}[h]
					\centering
					\caption{MLX90640の精度}
					\begin{tabular}{|c|r|}
						\hline	動作電圧	& 3~3.6V \\
						\hline	消費電流	&	23mA\\
						\hline	視野	&  110°×75°\\
						\hline	測定範囲	&	-40℃~+300℃\\
						\hline	分解能	& ±1.5℃\\
						\hline	リフレッシュレート	&	0.5Hz~64Hz\\
						\hline	解像度	&	32×24ピクセル\\
						\hline
					\end{tabular}
					\label{MLX90640}
				\end{table}		
				