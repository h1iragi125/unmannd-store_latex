\graphicspath{{./contents/chapter1/figures/}}	% 図・画像ファイルが保存されたディレクトリまでのパス
\chapter{序論}
	\section{研究背景}
		近年,様々な技術の発達に伴い,人々の生活はより快適になっている.その技術として,デジタル技術が挙げられる.これには,家電製品や自動車,センサ,カメラなどの様々な「モノ」がインターネットに接続し,相互にデータをやり取りするIoT(Internet of Things)$^{\cite{live}}$,感知器や測定器などを用いて対象の定量的な情報を取得するセンシング$^{\cite{live}}$,コンピューターに人間の知的活動を模倣させるAI(Artificial Intelligence)などが該当し$^{\cite{live}}$,農業,製造業,医療,物流と幅広い分野で活用が進んでいる$^{\cite{live}}$.
 具体的な活用事例として, スマートストアが挙げられる.これは,デジタル技術を取り入れ,様々な作業を自動化,最適化した小売店舗のことである. スマートストアでは,決済処理や商品管理,混雑状況の把握,入退店管理などにデジタル技術が活用されており,効率的で正確な店舗運営の実現に寄与している[3].
 このように,デジタル技術の普及により,データの収集および活用が容易となり,業務の効率化やシステムの高度化が進んでいる.そして, 大学においてもこのような技術の需要は高まっている.
 大学では,キャンパス内に設置された購買施設の1つとして,\figref{Store}に示す無人店舗が存在する.この施設は,学生をはじめとする多くの人々に利用されており,学内コミュニティの活性化に寄与している.その一方,無人店舗の運営においていくつかの課題が存在し,特に重要な課題として,在庫状況を手軽に確認する手段が十分に整備されていない点が挙げられる.これにより,目当ての商品を購入するために無人店舗を訪れたにもかかわらず,売り切れや欠品によって商品を入手できず,時間を浪費してしまう可能性がある.また,無人店舗から離れた場所では,販売されている商品を確認できないことから,無人店舗が利用選択肢として選ばれにくくなり,結果として無人店舗および利用者の双方に悪影響を及ぼすことが懸念される.そこで,無人店舗の在庫状況を遠隔から把握可能なシステムが導入されれば,無人店舗の利便性向上による利用者の増加や,さらなる学生コミュニティの活性化が期待できる.
		\vspace{0.5\baselineskip}
		\begin{figure}[H]
			\centering
			\includegraphics[width=.5\linewidth]{図1.1.jpg}
			\caption{大学内の無人店舗}
			\label{Store}
		\end{figure}
		
	\section{研究目的}
		本研究では,大学キャンパス内に設置された無人店舗に超音波センサやデプスカメラを設置し,在庫状況に関するデータを収集,管理する方法および収集したデータを利用者に共有する方法を検討する.さらに,これらの検討結果に基づき,利用者が遠隔から在庫状況を把握可能なシステムの構築を目的とする.

		
		