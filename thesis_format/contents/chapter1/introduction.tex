\graphicspath{{./contents/chapter1/figures/}}	% 図・画像ファイルが保存されたディレクトリまでのパス
\chapter{序論}
	\section{研究背景}
	近年,様々な技術の発達に伴い,人々の生活はより快適になっている.その中でも,IT(Information Technology)技術は,コンピューターおよびネットワークを利用して情報の処理や通信を行う技術の総称であり,社会的重要性が高まっている.例えば,画像データに対し情報の変換や特徴量の抽出などの処理を行い,目的に応じた分析および判定を可能にする画像処理\cite{1.6},感知器や測定器などを用いて対象の定量的な情報を取得するセンシング\cite{1.2}が挙げられる.これらの技術は,農業,製造業,医療,物流など幅広い分野で活用されており,作業の効率化や生産性の向上に寄与している.
	
	画像処理およびセンシングの活用事例として,\figref{Store}に示す成蹊大学11号館1階に設置された無人店舗が挙げられる.この施設では,複数のAI(Artificial Intelligence)カメラおよびセンサを用いて利用者の動作検知や商品の識別を行うことで,キャッシュレス決済やデータの収集,リアルタイム処理を実現している.これにより,利用者の円滑な購買行動を促進し,効率的な店舗運営が可能になる\cite{1.7}.その一方,無人店舗の運営においていくつかの課題が存在する.主要な課題の1つとして,AIカメラおよびセンサによって収集されたデータを利用する独自のシステム構築が困難である点が挙げられる.これらのデータは無人店舗を管理している企業が保有しており,成蹊大学側に提供されていないため,容易に入手することができない.また別の主要な課題として,地理的条件に加え,訪問前に在庫状況を把握できないことにより,無人店舗が利用選択肢として選ばれづらい点が挙げられる.\figref{map}に示す成蹊大学キャンパスマップから分かる通り無人店舗はキャンパスの端に近い場所に設置されており,正門からの距離も大きく離れている.そのため,店舗を訪れたにもかかわらず,目当ての商品が売り切れや欠品によって入手できない場合,利用者が時間を浪費してしまう可能性がある.そこで独自にデータを収集,利用し,在庫状況を遠隔から手軽に確認するシステムを構築することで,利用可能なデータの入手方法の確立および無人店舗の利便性向上による利用者の増加が期待できる.

	\vspace{0.5\baselineskip}
	\begin{figure}[H]
		\centering
		\includegraphics[width=.5\linewidth]{図1.1.jpg}
		\caption{大学内の無人店舗}
		\label{Store}
	\end{figure}

	\vspace{0.5\baselineskip}
	\begin{figure}[H]
		\centering
		\includegraphics[width=.8\linewidth]{図1.2.png}
		\caption{成蹊大学キャンパスマップ}
		\label{map}
	\end{figure}
		
	\section{研究目的}
		本研究では,成蹊大学キャンパス内に設置された無人店舗を対象とし,OpenCVを用いた画像処理,超音波センサを用いたセンシング,デプスカメラを用いたセンシングという3つの手法から在庫状況に関するデータの収集を行う.さらに,各手法の利点と欠点を整理し,データ収集の精度および設置の容易性の観点から実現性の最も高い手法を検討する.加えて,収集したデータをLINE Botを用いて利用者に共有することで,遠隔から在庫状況を把握可能なシステムの構築を目的とする.
		
		