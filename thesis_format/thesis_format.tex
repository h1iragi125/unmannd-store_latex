%\RequirePackage[2023/04/01]{latexrelease}		% TeX Live環境を推奨
\documentclass[lualatex,book,openany,oneside,jafontsize=10pt,jafontscale=0.92469,jlreq_warnings]{jlreq}

%---------- PREAMBLE ---------- 
% スタイルファイルの読み込み(オプション:thebibenv->thebibliography環境,usebiber->biblatex+biber)
\usepackage[thebibenv]{./labstyle/thesismacro}	% プロジェクト内を指定する場合

% 参考文献の設定の読み込み(biblatex+biberによる作成の場合のみ->.styのオプションにに"usebiber"を指定)
%\addbibresource{./contents/reference/reference.bib}	% 参考文献リスト(.bib)を指定(biberの実行が必要)

% 和題("\\"で任意の改行位置を設定可能)
\titleJP{論文フォーマット\\日本語タイトル}
% 英題("\\"で任意の改行位置を設定可能)
\titleEN{Thesis Format\\English Title}
% 著者の学年(B4卒業論文->B4,修士論文->M2)
\grade{B4}
% 著者の人数(単著->1,共著->2 or 3)
\authorsnum{2}
% "\(n th)auther{student ID}{lastname}{firstname}"->(不要な箇所は空白 or コメントアウト・削除)
\firstauthor{M222221}{今井}{直幸}
\secondauthor{M222222}{今井}{直太郎}
%\thirdauthor{M222222}{石塚}{伶}
% 指導教員名
\advisor{小川 隆申}

\calcfiscalyear{\year}{\month}				% 年度を出力("\year","\month"でビルドした日の年度を計算)
\filingdate{\year}{\month}{\day}			% 提出日を記入("year","\month","\day"でビルドした日)
\modified{0}								% 修正回数(2回までしか出力されない)
\firstmodifiddate{\year}{9}{4}				% 第一次修正日
\secondmodifiddate{1999}{\month}{\day}		% 第二次修正日

% PDFメタデータの設定(任意)
\hypersetup{%
	pdftitle={論文フォーマット},				% PDFタイトル
	pdfsubject={template},					% PDFサブタイトル
	pdfauthor={Naoyuki Imai},				% 著者
	pdfkeywords={LuaLaTeX,template},		% キーワード
}

\begin{document}
	%\layout								% ページレイアウトの確認用ページ出力コマンド
	\begin{titlepage}
		%\thispagestyle{empty}
		\fluidlabtitleformat				% 論文表紙:流体研フォーマット(プリアンブルで各種変数を設定)
	\end{titlepage}
	
	%\pagestyle{plain}						% "empty"でページ番号off
	
	\frontmatter		% \chapter毎に\phantomsectionの設定が必要(これでハイパーリンクを設定)
	% --- 記号・略語一覧
	\phantomsection\chapter{記号一覧}
\begin{tabular}{lcl}
	% $v$ & : & 速度 \\
	% $p$ & : & 圧力 \\
	% $\nu$ & : & 動粘性係数 \\
	% $\mu$ & : & 粘性係数 \\
\end{tabular}
	
	% ---------- TABLE OF CONTENTS ----------
	\tableofcontents	% 目次の表示
	
	\mainmatter			% \phantomsectionは不要
	% ---------- CONTENTS ----------
	%\include{ファイル名} \setcounter{equation}{0}:数式番号初期化 \setcounter{figure}{0}:図表番号初期化
	
	% --- 序論(chapter1)
	\graphicspath{{./contents/chapter1/figures/}}	% 図・画像ファイルが保存されたディレクトリまでのパス
\chapter{序論}
	\section{研究背景}
		近年,様々な技術の発達に伴い,人々の生活はより快適になっている.その技術として,デジタル技術が挙げられる.これには,家電製品や自動車,センサ,カメラなどの様々な「モノ」がインターネットに接続し,相互にデータをやり取りするIoT(Internet of Things)$^{\cite{live}}$,感知器や測定器などを用いて対象の定量的な情報を取得するセンシング$^{\cite{live}}$,コンピューターに人間の知的活動を模倣させるAI(Artificial Intelligence)などが該当し$^{\cite{live}}$,農業,製造業,医療,物流と幅広い分野で活用が進んでいる$^{\cite{live}}$.
 具体的な活用事例として, スマートストアが挙げられる.これは,デジタル技術を取り入れ,様々な作業を自動化,最適化した小売店舗のことである. スマートストアでは,決済処理や商品管理,混雑状況の把握,入退店管理などにデジタル技術が活用されており,効率的で正確な店舗運営の実現に寄与している[3].
 このように,デジタル技術の普及により,データの収集および活用が容易となり,業務の効率化やシステムの高度化が進んでいる.そして, 大学においてもこのような技術の需要は高まっている.
 大学では,キャンパス内に設置された購買施設の1つとして,\figref{Store}に示す無人店舗が存在する.この施設は,学生をはじめとする多くの人々に利用されており,学内コミュニティの活性化に寄与している.その一方,無人店舗の運営においていくつかの課題が存在し,特に重要な課題として,在庫状況を手軽に確認する手段が十分に整備されていない点が挙げられる.これにより,目当ての商品を購入するために無人店舗を訪れたにもかかわらず,売り切れや欠品によって商品を入手できず,時間を浪費してしまう可能性がある.また,無人店舗から離れた場所では,販売されている商品を確認できないことから,無人店舗が利用選択肢として選ばれにくくなり,結果として無人店舗および利用者の双方に悪影響を及ぼすことが懸念される.そこで,無人店舗の在庫状況を遠隔から把握可能なシステムが導入されれば,無人店舗の利便性向上による利用者の増加や,さらなる学生コミュニティの活性化が期待できる.
		\vspace{0.5\baselineskip}
		\begin{figure}[H]
			\centering
			\includegraphics[width=.5\linewidth]{図1.1.jpg}
			\caption{大学内の無人店舗}
			\label{Store}
		\end{figure}
		
	\section{研究目的}
		本研究では,大学キャンパス内に設置された無人店舗に超音波センサやデプスカメラを設置し,在庫状況に関するデータを収集,管理する方法および収集したデータを利用者に共有する方法を検討する.さらに,これらの検討結果に基づき,利用者が遠隔から在庫状況を把握可能なシステムの構築を目的とする.

		
		
	
	% --- ex)解析手法(chapter2)
	\graphicspath{{./contents/chapter2/figures/}}	% 図・画像ファイルが保存されたディレクトリまでのパス
\chapter{解析手法}
	\section{サーマルカメラ}
		本研究では、混雑状況の測定に\figref{Camera}のサーマルカメラで撮影された赤外線画像を用いる。このサーマルカメラはMLX90640赤外線アレイモジュールを搭載したM5Stack用赤外線画像ユニットであり、小型のマイコンボードであるM5Stackに接続することにより撮影を行うことができる。
		
		\vspace{0.5\baselineskip}
		\begin{figure}[h]
			\centering
			\includegraphics{MLX90640.png}
			\caption{Melexis社 MLX90640}
			\label{Camera}
		\end{figure}
	
			\subsection{サーマルカメラの精度}
				MLX90640の性能を\tabref{MLX90640}に示す。測定範囲は-40℃~+300℃であり、混雑状況のモニタリングの際に観測される温度は測定可能であると考えられる。
				
				\vspace{0.5\baselineskip}
				\begin{table}[h]
					\centering
					\caption{MLX90640の精度}
					\begin{tabular}{|c|r|}
						\hline	動作電圧	& 3~3.6V \\
						\hline	消費電流	&	23mA\\
						\hline	視野	&  110°×75°\\
						\hline	測定範囲	&	-40℃~+300℃\\
						\hline	分解能	& ±1.5℃\\
						\hline	リフレッシュレート	&	0.5Hz~64Hz\\
						\hline	解像度	&	32×24ピクセル\\
						\hline
					\end{tabular}
					\label{MLX90640}
				\end{table}		
				
	
	% --- ex)結果(chapter3)
	\graphicspath{{./contents/chapter3/figures/}}	% 図・画像ファイルが保存されたディレクトリまでのパス
\chapter{結果}
	結果である。
	
	
	% --- 結論(chapter4)
	\graphicspath{{./contents/chapter7/figures/}}	% 図・画像ファイルが保存されたディレクトリまでのパス
\chapter{総合的考察}
	本章では、第3章で実施した超音波センサによる在庫把握と、第4章で実施したデプスカメラによる在庫把握、、について考察を行う。
	\section{本研究の限界と今後の課題}
	本節では、本研究における限界を明示するとともに、今後の課題について述べる。
		\subsection{本研究の課題}

			\subsubsection{超音波センサに関する課題}
			まず、超音波センサでの在庫数把握において、商品が購入されたあと、在庫が
			一部の小売店では、に

			在庫把握には、装置を大量に用意
	
	\backmatter			% \chapter毎に\phantomsectionの設定が必要(これでハイパーリンクを設定)
	
	% --- 参考文献
	% --- thebibliography環境->preambleで"\usepackage[thebibenv]{thesismacro}"を指定
	\begin{thebibliography}{99} % 二桁分の番号幅を確保
	\setlength{\itemsep}{6pt}	% 文献毎に行間を少し開けるコマンド(不要であればコメントアウト)
		
	% 参考文献定形
	% \bibitem{} ,「」,\url{},最終更新

	% chapter1
	% \bibitem{1.1} SpaceCore,「IoTとは?わかりやすく簡単に解説。仕組みや活用法、課題も紹介」,\url{https://space-core.jp/media/14453}/,最終更新2024.09.17
	\bibitem{1.2} Degital Intelligence チャンネル,「センシングとは? 基礎から注意点を分かりやすく解説し活用例を紹介」,\url{https://www.cloud-for-all.com/blog/what-is-sensing},最終更新2024.12.11
	% \bibitem{1.3} AI総合研究所,「AIとは何か?仕組み・種類・技術用語まで完全ガイド【2025年版】」,\url{https://www.ai-souken.com/article/ai-other},最終更新2025.04.16
	% \bibitem{1.4} Rakuten Mobile,「IoTとは?読み方・意味・仕組みや活用事例を簡単に紹介」,\url{https://network.mobile.rakuten.co.jp/sumakatsu/contents/articles/2025/00458/?msockid=1677875e543f61522301935c554a60a2},最終更新2025.12.10
	% \bibitem{1.5} TTG,「スマートストアとは?仕組みやメリット、国内の事例を徹底解説」,\url{https://ttg.co.jp/media/what-is-smart-store/},最終更新2025.06.10
	\bibitem{1.6} Schoo for Business,「画像処理とは?その特徴や活用される事例について解説する」,\url{https://schoo.jp/biz/column/885},最終更新2025.09.19
	\bibitem{1.7} TTG,「無人店舗の仕組みとは?メリットやデメリットと万引き対策まで徹底解説」,\url{https://ttg.co.jp/media/unmanned-store-system},最終更新2025.05.08

	% chapter2
	\bibitem{2.1} Device HD,「Sony Xperia 10 VI 5G」,\url{https://devicehd.com/smartphones/en/product/66c3181893f2fb776375368c/}
	\bibitem{2.2} SAMURAIENGINNER Blog,「OpenCVとは?できることや特徴をわかりやすく解説」,https://www.sejuku.net/blog/113292,最終更新2025.12.26
	\bibitem{2.3} Qiita,「グレースケール画像のうんちく」,\url{https://qiita.com/yoya/items/96c36b069e74398796f3},最終更新2025.04.20
	\bibitem{2.4} MUSASHI AI,「グレースケール変換」,\url{https://musashi-ai.com/glossary/2023/06/132b15f51be239030f34b40a152b0c724506964c}
	\bibitem{2.5} Mustafa Murat ARAT,「RGB to Grayscale Conversion」,\url{https://mmuratarat.github.io/2020-05-13/rgb_to_grayscale_formulas},最終更新2020.05.13
	\bibitem{2.6} Gigahertz-Optik,「1.6 Spectral Sensitivity of the Human Eye」,\url{https://www.gigahertz-optik.com/en-us/service-and-support/knowledge-base/basics-light-measurement/light-color/spectr-sens-eye/}
	\bibitem{2.7} Britannica,「RGB color model」,\url{https://www.britannica.com/science/RGB-color-model},最終更新2026.01.23
	\bibitem{2.8} PEKO STEP,「HSV色空間」,\url{https://www.peko-step.com/html/hsv.html}
	\bibitem{2.9} OpenCV,「Color spaces in OpenCV」,\url{https://opencv.org/blog/color-spaces-in-opencv/#h-hsv-hue-saturation-value-color-space},最終更新2025.04.29
	\bibitem{2.10} 数理超入門部,「HSV色空間とは?RGBから変換するときの計算式」,\url{https://algorithm.joho.info/image-processing/hsv-color-space/},最終更新2017.07.04
	\bibitem{2.11} IT用語辞典 e-Words,「閾値【threshold】しきい値」,\url{https://e-words.jp/w/%E9%96%BE%E5%80%A4.html}
	\bibitem{2.12} OpenCV オープンソースのすすめ,「OpenCVのfindContours関数を使った画像の輪郭検出」,\url{https://www.argocorp.com/OpenCV/imageprocessing/opencv_find_contours.html}

	% chapter3
	\bibitem{3.1} フェイシングスタンド 河淳 スタンド/仕切板/仕切ワイヤー 【通販モノタロウ】,\url{https://www.monotaro.com/g/02794558/}, 2026/01/29閲覧

	% chapter4
	\bibitem{4.1} ArduCAM/Arducam\_tof\_camera,\url{https://github.com/ArduCAM/Arducam_tof_camera}, 2026/01/29閲覧
	\bibitem{4.2} ToF特集 ToFカメラとは? ToFカメラを使ってできること|inrevium \url{https://www.inrevium.com/pickup/tofcamera/}, 2026/01/29閲覧
	\bibitem{4.3} Arducam ToF Camera SDK – for Raspberry Pi - Arducam Wiki \url{https://docs.arducam.com/Raspberry-Pi-Camera/Tof-camera/Arducam-ToF-Camera-SDK/}
	% chapter5
	\bibitem{5.1} note,「LINE Botとは何か?初心者向けにわかりやすく解説」,\url{https://note.com/bonjour_maman/n/n2a2ce62cde15},最終更新2025.09.16
	\bibitem{5.2} 総務省情報通信政策研究所,「令和6年度情報通信メディアの利用時間と情報行動に関する調査報告書(概要)」p.12,\url{https://www.soumu.go.jp/main_content/001017240.pdf},最終更新2025.06
	\bibitem{5.3} FirstContact,「Messaging APIとは?~意味やできることを解説!~」,\url{https://first-contact.jp/blog/article/messaging-api/},最終更新2025.03.05
	\bibitem{5.4} Qiita,「初心者向け解説:APIとは?その仕組みと活用法を分かりやすく解説」,\url{https://qiita.com/UKI_datascience/items/18605ce56c7d9a4e4ca0},最終更新2025.01.16
	\bibitem{5.5} ミライサーバー,「Flaskとは?基本知識からインストール手順まで詳しく解説!」,\url{https://www.miraiserver.ne.jp/column/about_flask/},最終更新2025.08.18
	\bibitem{5.6} blastengine,「Webhookとは?仕組みやメリット、APIとの違い、利用方法について分かりやすく解説」,\url{https://blastengine.jp/blog_content/webhook/},最終更新2024.10.31
	\bibitem{5.7} Qiita,「Ngrokの使い方・実際に私が使っている事例を紹介」,\url{https://qiita.com/halapolo/items/a9d2345836b0302a264d},最終更新2025.05.14
	\bibitem{5.8} iifx.dev,「Flask開発サーバーをネットワーク公開する完全ガイド」,\url{https://iifx.dev/ja/articles/32624714/flask%E9%96%8B%E7%99%BA%E3%82%B5%E3%83%BC%E3%83%90%E3%83%BC%E3%82%92%E3%83%8D%E3%83%83%E3%83%88%E3%83%AF%E3%83%BC%E3%82%AF%E5%85%AC%E9%96%8B%E3%81%99%E3%82%8B%E5%AE%8C%E5%85%A8%E3%82%AC%E3%82%A4%E3%83%89},最終更新2025.07.19
	\bibitem{5.9} LINE Developers,「Messaging APIの料金」,\url{https://developers.line.biz/ja/docs/messaging-api/pricing/}
	\bibitem{5.10} LINE Developers,「リッチメニューの概要」,\url{https://developers.line.biz/ja/docs/messaging-api/rich-menus-overview/}

	% chapter6

	% chapter7

\end{thebibliography}
	% --- biblatex+biber->preambleで"\usepackage[usebiber]{thesismacro}"&"\addbibresource{.bib}"を指定
	%\printbibliography	% 文献リストの挿入
	
	% --- 謝辞
	\phantomsection\chapter{謝辞}
	本研究にご協力ならびにご助言をいただいた全ての皆様に深く感謝いたします。特に、小川隆申教授ならびに謝文昂助教には、終始熱心なご指導をいただきました。併せて、流体力学研究室の皆様にはご助言とご支援をいただきましたことに深く感謝いたします。
	
	\mainmatter			% \phantomsectionは不要
	% ---------- APPENDIX ----------
	\appendix	% 以下は付録
	% --- 関連プログラム
	% appendixで示すコードののディレクトリパス(絶対パスが使えなくなる & 一度に1つしか指定できないので注意)
\lstset{inputpath=./contents/appendix/source_code/program_files/}
\chapter{プログラム}

	\section{ソースコード}
		以下はソースコードである。超音波センサでの在庫数把握において、マイコンボードで実行したプログラムが\ref{measurepy}である。
		\hypertarget{measurepy}{% ハイパーリンク先の設定\hyperlink{label}{}でここに飛ぶ(内部のlabelと合わせると便利かも)
			\lstinputlisting[style=mypystyle,caption=findContours,label=measurepy]{measure.py}
		}
		超音波センサ
		\ref{machine}に機械学習に用いた一連のプログラムを示す。
		\hypertarget{machine}{% ハイパーリンク先の設定\hyperlink{label}{}でここに飛ぶ(内部のlabelと合わせると便利かも)
			\lstinputlisting[style=mypystyle,caption=MachineLearning,label=machine]{countpeople.py}
		}
	
	% --- 原稿非掲載データ
	\chapter{原稿非掲載データ}
	\section{実験結果}
		例えばここに本文中に載せられなかった実験結果などを載せる.

\end{document}