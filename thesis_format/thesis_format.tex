%\RequirePackage[2023/04/01]{latexrelease}		% TeX Live環境を推奨
\documentclass[lualatex,book,openany,oneside,jafontsize=10pt,jafontscale=0.92469,jlreq_warnings]{jlreq}

%---------- PREAMBLE ---------- 
% スタイルファイルの読み込み(オプション:thebibenv->thebibliography環境,usebiber->biblatex+biber)
\usepackage[thebibenv]{./labstyle/thesismacro}	% プロジェクト内を指定する場合

% 参考文献の設定の読み込み(biblatex+biberによる作成の場合のみ->.styのオプションにに"usebiber"を指定)
%\addbibresource{./contents/reference/reference.bib}	% 参考文献リスト(.bib)を指定(biberの実行が必要)

% 和題("\\"で任意の改行位置を設定可能)
\titleJP{IoTを用いた無人店舗における\\[-1.0em]遠隔在庫把握システムの構築}
% 英題("\\"で任意の改行位置を設定可能)
\titleEN{Development of a Remote Inventory Monitoring System \\for Unmanned Retail Stores Using IoT}
% 著者の学年(B4卒業論文->B4,修士論文->M2)
\grade{B4}
% 著者の人数(単著->1,共著->2 or 3)
\authorsnum{2}
% "\(n th)auther{student ID}{lastname}{firstname}"->(不要な箇所は空白 or コメントアウト・削除)
\firstauthor{S226109}{前澤}{柊飛}
\secondauthor{S226003}{秋元}{大生}
%\thirdauthor{M222222}{石塚}{伶}
% 指導教員名
\advisor{小川 隆申}

\calcfiscalyear{\year}{\month}				% 年度を出力("\year","\month"でビルドした日の年度を計算)
\filingdate{\year}{\month}{\day}			% 提出日を記入("year","\month","\day"でビルドした日)
\modified{0}								% 修正回数(2回までしか出力されない)
\firstmodifiddate{\year}{9}{4}				% 第一次修正日
\secondmodifiddate{1999}{\month}{\day}		% 第二次修正日

% PDFメタデータの設定(任意)
\hypersetup{%
	pdftitle={IoTを用いた無人店舗における遠隔在庫把握システムの構築},				% PDFタイトル
	pdfsubject={Development of a Remote Inventory Monitoring System for Unmanned Retail Stores Using IoT}					% PDFサブタイトル
	pdfauthor={前澤 柊飛, 秋元 大生}				% 著者
	pdfkeywords={IoT,Stores,店舗,Remote,遠隔,Monitoring,在庫把握}		% キーワード
}

\begin{document}
	%\layout								% ページレイアウトの確認用ページ出力コマンド
	\begin{titlepage}
		%\thispagestyle{empty}
		\fluidlabtitleformat				% 論文表紙:流体研フォーマット(プリアンブルで各種変数を設定)
	\end{titlepage}
	
	%\pagestyle{plain}						% "empty"でページ番号off
	
	\frontmatter		% \chapter毎に\phantomsectionの設定が必要(これでハイパーリンクを設定)
	% --- 記号・略語一覧
	\phantomsection\chapter{記号一覧}
\begin{tabular}{lcl}
	% $v$ & : & 速度 \\
	% $p$ & : & 圧力 \\
	% $\nu$ & : & 動粘性係数 \\
	% $\mu$ & : & 粘性係数 \\
\end{tabular}
	
	% ---------- TABLE OF CONTENTS ----------
	\tableofcontents	% 目次の表示
	
	\mainmatter			% \phantomsectionは不要
	% ---------- CONTENTS ----------
	%\include{ファイル名} \setcounter{equation}{0}:数式番号初期化 \setcounter{figure}{0}:図表番号初期化
	
	% --- 序論(chapter1)
	\graphicspath{{./contents/chapter1/figures/}}	% 図・画像ファイルが保存されたディレクトリまでのパス
\chapter{序論}
	\section{研究背景}
		近年,様々な技術の発達に伴い,人々の生活はより快適になっている.その技術として,デジタル技術が挙げられる.これには,家電製品や自動車,センサ,カメラなどの様々な「モノ」がインターネットに接続し,相互にデータをやり取りするIoT(Internet of Things)$^{\cite{live}}$,感知器や測定器などを用いて対象の定量的な情報を取得するセンシング$^{\cite{live}}$,コンピューターに人間の知的活動を模倣させるAI(Artificial Intelligence)などが該当し$^{\cite{live}}$,農業,製造業,医療,物流と幅広い分野で活用が進んでいる$^{\cite{live}}$.
 具体的な活用事例として, スマートストアが挙げられる.これは,デジタル技術を取り入れ,様々な作業を自動化,最適化した小売店舗のことである. スマートストアでは,決済処理や商品管理,混雑状況の把握,入退店管理などにデジタル技術が活用されており,効率的で正確な店舗運営の実現に寄与している[3].
 このように,デジタル技術の普及により,データの収集および活用が容易となり,業務の効率化やシステムの高度化が進んでいる.そして, 大学においてもこのような技術の需要は高まっている.
 大学では,キャンパス内に設置された購買施設の1つとして,\figref{Store}に示す無人店舗が存在する.この施設は,学生をはじめとする多くの人々に利用されており,学内コミュニティの活性化に寄与している.その一方,無人店舗の運営においていくつかの課題が存在し,特に重要な課題として,在庫状況を手軽に確認する手段が十分に整備されていない点が挙げられる.これにより,目当ての商品を購入するために無人店舗を訪れたにもかかわらず,売り切れや欠品によって商品を入手できず,時間を浪費してしまう可能性がある.また,無人店舗から離れた場所では,販売されている商品を確認できないことから,無人店舗が利用選択肢として選ばれにくくなり,結果として無人店舗および利用者の双方に悪影響を及ぼすことが懸念される.そこで,無人店舗の在庫状況を遠隔から把握可能なシステムが導入されれば,無人店舗の利便性向上による利用者の増加や,さらなる学生コミュニティの活性化が期待できる.
		\vspace{0.5\baselineskip}
		\begin{figure}[H]
			\centering
			\includegraphics[width=.5\linewidth]{図1.1.jpg}
			\caption{大学内の無人店舗}
			\label{Store}
		\end{figure}
		
	\section{研究目的}
		本研究では,大学キャンパス内に設置された無人店舗に超音波センサやデプスカメラを設置し,在庫状況に関するデータを収集,管理する方法および収集したデータを利用者に共有する方法を検討する.さらに,これらの検討結果に基づき,利用者が遠隔から在庫状況を把握可能なシステムの構築を目的とする.

		
		
	
	% --- 画像処理によるデータ収集(chapter2)
	\graphicspath{{./contents/chapter2/figures/}}	% 図・画像ファイルが保存されたディレクトリまでのパス
\chapter{画像処理による在庫検出}
無人店舗の在庫状況に関するデータの収集を実現するため,複数の技術的アプローチについて検討を行い,最適な在庫検出手法を明確にする.本章では,画像処理技術を用いた在庫検出について,OpenCVを用いた検出方法,輪郭抽出,座標による商品識別,検出した結果と考察を説明する.
	\section{画像取得に用いた機器}
		画像処理を行うにあたり,無人店舗の商品棚の画像を取得するため,スマートフォンのカメラ機能を利用した.本章では,画像処理技術を用いた在庫検出の実現性について検討することを主な目的としているため,コスト削減や画像取得の容易さの観点からこの選択をした.実際の画像取得に用いたスマートフォンを\figref{Camera}に,取得した商品棚の画像を\figref{onigiri},\figref{colorful}に示す.
		% [width=.5\linewidth]
		\vspace{0.5\baselineskip}
		\begin{figure}[h]
			\centering
			\includegraphics[width=.4\linewidth]{図2.1.png}
			\caption{画像取得用スマートフォン\cite{2.1}}
			\label{Camera}
		\end{figure}

		\vspace{0.5\baselineskip}
		\begin{figure}[H]
			\centering
			\begin{minipage}{.45\linewidth}
				\centering
				\includegraphics[width=\linewidth]{図2.2.jpg}
				\caption{おにぎりを中心とする商品棚}
				\label{onigiri}
			\end{minipage}
			\hfill
			\vspace{0.5\baselineskip}
			\begin{minipage}{.45\linewidth}
				\centering
				\includegraphics[width=\linewidth]{図2.3.png}
				\caption{様々な色の商品が並ぶ棚}
				\label{colorful}
			\end{minipage}
		\end{figure}

		画像を一定の品質で取得するため,カメラは等倍かつ品棚から約3mの位置で撮影を行う.\figref{shelf}に示す一部結果を可視化した画像において,商品の周りに検出を表現する矩形(バウンディングボックス)が表示されていることから,本章で用いたスマートフォンのカメラ機能は,画像処理による商品検出に十分な性能を満たしていると判断した.スマートフォンのカメラ性能を\tabref{smartphone}に示す.
				
		\vspace{0.5\baselineskip}
		\begin{figure}[h]
			\centering
			\includegraphics[width=.7\linewidth]{図2.4.png}
			\caption{一部商品棚の検出結果}
			\label{shelf}
		\end{figure}

		\vspace{0.5\baselineskip}
		\begin{table}[h]
			\centering
			\caption{スマートフォンのカメラ性能}
			\begin{tabular}{|c|r|}
				\hline	使用機種	& Sony Xperia 10 VI 5G \\
				\hline	使用カメラ  & メインカメラ\\
				\hline	画像解像度 & 1920×1080\\
				\hline	総画素数 & 約207万画素\\
				\hline
			\end{tabular}
			\label{smartphone}
		\end{table}		
				
	\tabref{smartphone}に示した使用するカメラの画像解像度が1920×1080であることから,本章は,解像度1920×1080と同様,もしくはそれ以上の性能を有するカメラを用いたときの画像処理技術による在庫検出について検討する.

	\section{OpenCV}
	OpenCV(Open Source Computer Vision Library)は,画像および動画に関する処理機能・検出機能をまとめたオープンソースのライブラリである\cite{2.2}. WindowsやLinux,iOS,AndroidなどさまざまなOSに対応しており,Raspberry Piなどの端末上で利用することもできる.さらに,豊富な画像処理機能を搭載しており,高度な画像処理を比較的容易に実装できるという特徴を有しているため,本研究の目的達成に重要な,様々な種類の商品の高精度検出およびリアルタイム性のあるデータ収集に最適だと考え,今回OpenCVを使用した.
		
		\subsection{グレースケール変換}
		OpenCVの機能にグレースケール変換がある.これは,色の情報を省き,明るさの度合いという情報のみで表現するための画像処理であり\cite{2.3},\figref{gray}に示すグレースケール変換前後の画像比較から分かる通り,グレースケール変換によって,画像内の明暗を明確にすることができる.

		\vspace{0.5\baselineskip}
		\begin{figure}[H]
			\centering
			\includegraphics[width=.8\linewidth]{図2.5.png}
			\caption{グレースケール変換を適用した画像}
			\label{gray}
		\end{figure}

		通常,処理を行う前のカラー画像(以下RGB画像とする)は,R(Red),G(Green),B(Blue)の3つの値(以下RGB値とする)で表現され,それぞれ0~255の数値をとる.一方,グレースケール変換を行った画像(以下グレースケール画像とする)は,輝度という1つの値で表現され \cite{2.4},0~255の数値をとる.\figref{kido}に示す輝度の値による明るさの違いから分かる通り,値が小さいと暗く,大きいと明るく表現される.

		\vspace{0.5\baselineskip}
		\begin{figure}[H]
			\centering
			\includegraphics[width=.7\linewidth]{図2.6.png}
			\caption{輝度の値による明るさ}
			\label{kido}
		\end{figure}

		グレースケール変換の仕組みについて説明する.変換方法は複数存在するが,OpenCVでは,加重平均法が用いられている\cite{2.5}.グレースケール画像を表現する輝度という値は,RGB画像のRGB値から計算された値であり,加重平均法はR,G,Bの値それぞれに重み付けをして輝度を計算する方法である.
		\noindent OpenCVにおいて輝度は

		\begin{equation} % equation環境(式番号付き0)
			輝度 = 0.2989 \times R + 0.5870 \times G + 0.1140 \times B
			\label{kidosiki}
		\end{equation}
		
		\noindent である.このRGBそれぞれに対する重みは,緑の光が最も検知しやすく,青の光は比較的検知しにくいという人間の視覚特性をもとに設定されている\cite{2.6}.

		グレースケール変換の活用が有効な場面を検討する. RGBはデジタル機器同様,人間の視覚が光の総量に強く依存する性質を利用して構造化されており\cite{2.7},光の三原色として表現される.そのため,RGB値をすべて255にして表現される色は白であり,RGB値をすべて0にして表現される色は黒である.\eqref{kidosiki}より,RGB値をすべて255にしたとき,輝度の値は最大である255をとり,RGB値をすべて0にしたとき,輝度の値は最小である0をとる.したがって,画像にグレースケール変換を適用した際,白に近い色を持つ物体は明るく表現され、黒に近い色を持つ物体は暗く表現される.
		本研究において,\figref{white}に示す実際の商品棚の一部から分かる通り,研究対象である無人店舗の一部の商品棚は,おにぎりのような白に近い色が特徴的な商品を中心に扱っている.また,\figref{black}に示す実際の商品棚の全体から分かる通り,無人店舗の商品棚全体が黒いデザインとなっている.したがって,おにぎりを中心に扱う商品棚に関しては,グレースケール変換を活用することで,商品と背景の差が明確になり,精度の高い検出が期待できる.加えて,グレースケール画像が輝度という1つの値で表現できる特徴から,情報量の削減による処理の高速化が期待できる.

		\vspace{0.5\baselineskip}
		\begin{figure}[H]
			\centering
			\begin{minipage}{.4\linewidth}
				\centering
				\includegraphics[width=\linewidth]{図2.7.png}
				\caption{白に近い色を持つ商品が多い棚}
				\label{white}
			\end{minipage}
			\hfill
			\vspace{0.5\baselineskip}
			\begin{minipage}{.5\linewidth}
				\centering
				\includegraphics[width=\linewidth]{図2.8.png}
				\caption{無人店舗の商品棚全体}
				\label{black}
			\end{minipage}
		\end{figure}

		\subsection{HSV変換}
		OpenCVの機能にHSV変換がある.これは,RGB画像を色相(Hue),彩度(Saturation),明度(Value)の3つの要素で表現された画像(以下HSV画像とする)に変換する機能である\cite{2.8}.
		色相(Hue)について説明する.これは,具体的な色の種類を表す要素で,\figref{colorwheel}に示す実際のカラーホイールから分かる通り,0~359の値を用いて様々な色を表現する.しかし,OpenCVでは効率的な処理を目的として8bit画像を主な対象としているため,扱う数値を0~255の範囲で表現することが望ましい.そのため,\figref{OpenCVwheel}に示すOpenCV のカラーホイールから分かる通り,OpenCVは色相の値を1/2にスケーリングし,0~179の値を用いて色を表現する\cite{2.9}.

		\vspace{0.5\baselineskip}
		\begin{figure}[H]
			\centering
			\begin{minipage}{.45\linewidth}
				\centering
				\includegraphics[width=\linewidth]{図2.9.png}
				\caption{一般的な色相範囲}
				\label{colorwheel}
			\end{minipage}
			\hfill
			\vspace{0.5\baselineskip}
			\begin{minipage}{.4\linewidth}
				\centering
				\includegraphics[width=\linewidth]{図2.10.png}
				\caption{OpenCVの色相範囲}
				\label{OpenCVwheel}
			\end{minipage}
		\end{figure}

		彩度(Saturation)について説明する.これは,色の鮮やかさや濃さを表す要素で\cite{2.8},色相と異なり,数値に明確な範囲が存在しないため,OpenCVは0~255の範囲に値をスケーリングして表現する.\figref{saido}に示す数値ごとの彩度表現から分かる通り,彩度の値が255をとるとき色は最も鮮やかになり,0の値をとるとき最も鈍くなる.

		\vspace{0.5\baselineskip}
		\begin{figure}[H]
			\centering
			\includegraphics[width=.7\linewidth]{図2.11.png}
			\caption{彩度の値による変化}
			\label{saido}
		\end{figure}

		明度(Value)について説明する.これは,色の明るさを表す要素で\cite{2.8},彩度と同様に,数値に明確な範囲が存在しないため,OpenCVは0~255の範囲に値をスケーリングして表現する.\figref{value}に示す数値ごとの明度表現から分かる通り,明度の値が255をとるとき色は最も明るくなり,0の値をとるとき最も暗くなる.

		\vspace{0.5\baselineskip}
		\begin{figure}[H]
			\centering
			\includegraphics[width=.7\linewidth]{図2.12.png}
			\caption{明度の値による変化}
			\label{value}
		\end{figure}

		HSV変換の仕組みについて説明する.グレースケール変換同様,HSV変換においてもRGB値を基に計算されている.RGB値をそれぞれ$R$,$G$,$B$とし,これらを0~1の範囲で表現したものをそれぞれ$R’$,$G’$,$B’$($R’=\frac{R}{255}$,$G’=\frac{G}{255}$,$B’=\frac{B}{255}$)とする.また,RGBの最大値($max(R,G,B)$)を$C_{max}$,最小値($min(R,G,B)$)を$C_{min}$とし,その差($C_{max}-C_{min}$)を$∆$とする.
		\noindent 色相($H$)は,$C_{min}=R$のとき

		\begin{equation} % equation環境(式番号付き0)
			H = \frac{1}{2}(60 \times (\frac{B'-G'}{Δ}) + 180)
			\label{H1}
		\end{equation}

		\noindent であり,$C_{min}=G$のとき

		\begin{equation} % equation環境(式番号付き0)
			H = \frac{1}{2}(60 \times (\frac{R'-B'}{Δ}) + 300)
			\label{H2}
		\end{equation}

		\noindent であり,$C_{min}=B$のとき

		\begin{equation} % equation環境(式番号付き0)
			H = \frac{1}{2}(60 \times (\frac{G'-R'}{Δ}) + 60)
			\label{H3}
		\end{equation}

		\noindent である\cite{2.10}.また,OpenCVでは$∆=0$ のとき,$H=0$として定義される.

		\noindent 彩度($S$)は

		\begin{equation} % equation環境(式番号付き0)
			S = \frac{Δ}{C_{max}}
			\label{S}
		\end{equation}

		\noindent である.また,OpenCVでは$C_{max}=0$ のとき,$S=0$として定義される.
		
		\noindent 明度($V$)は

		\begin{equation} % equation環境(式番号付き0)
			V = C_{max}
			\label{V}
		\end{equation}

		\noindent である[13].

		HSVは,色を鮮やかさや明るさという直感的に分かりやすい要素で表現するため,原色の組み合わせで表現するRGBに比べ,各要素を変動させた場合の色の変化がイメージしやすく,細かな色の調整が簡単にできる.
		本研究において,\figref{colorful}に示した実際の商品棚の一部から分かる通り,無人店舗の一部の商品棚は,様々な色の商品を扱っている.そのため,輝度の値が商品によって異なり,\figref{colorgray}に示す様々な色の商品が並ぶ棚のグレースケール画像を用いた検出結果から分かる通り,検出精度が低くなる.そこでHSV画像を活用することで,安定した精度の検出が期待できる.


		\vspace{0.5\baselineskip}
		\begin{figure}[h]
			\centering
			\includegraphics[width=.6\linewidth]{図2.13.png}
			\caption{様々な色の商品が並ぶ棚のグレースケール検出}
			\label{colorgray}
		\end{figure}

	\section{輪郭抽出}
	商品の位置を特定するにあたり,グレースケール画像およびHSV画像から商品の輪郭抽出を行う.具体的には,商品と背景を判別するための閾値を設定し,これを満たす画素集合を輪郭として抽出する.閾値とは,判断の境目となる値のことで\cite{2.11},OpenCVでは,利用者が任意に設定できる.

		\subsection{グレースケール画像の輪郭抽出}
		グレースケール画像に適用する閾値を考える.2.2.1節で述べたように,グレースケール画像では,輝度という0~255の範囲をとる1つの値で表現されるため,閾値はこの値を用いて適用する.
		\figref{light}に示す照明が商品棚へ及ぼす影響から分かる通り,無人店舗の商品棚は4段ある棚の段数のうち,最上段のみ照明が強く当たる構造になっている.そのため,輝度の値で表現するグレースケール画像では,全体に同一の閾値を設定すると,最上段とその他の段で検出精度に差が生じる.この問題に対処するため,画像内座標を利用し,最上段とそれ以外の範囲を切り分け,最上段の閾値を180~255,それ以外の範囲を100~255にそれぞれ設定する.この閾値は,\figref{white}に示したグレースケール変換の適用する商品棚において,様々な閾値を試した結果,最も検出精度が高くなると判断した値である.照明の当たり方を考慮した閾値設定のイメージを\figref{ikiti}に示す.

		\vspace{0.5\baselineskip}
		\begin{figure}[H]
			\centering
			\includegraphics[width=.8\linewidth]{図2.14.png}
			\caption{照明の商品棚への影響}
			\label{light}
		\end{figure}

		\vspace{0.5\baselineskip}
		\begin{figure}[H]
			\centering
			\includegraphics[width=.8\linewidth]{図2.15.png}
			\caption{グレースケール画像の閾値設定のイメージ}
			\label{ikiti}
		\end{figure}

		\subsection{HSV画像の輪郭抽出}
		HSV画像に適用する閾値を考える.2.2.2節で述べたように,HSV画像では,0~179の範囲をとる色相,0~255の範囲をとる彩度,明度という3つの値で表現されるため,閾値はこれらの値を用いて適用する.
		\figref{colorful}に示したHSV変換を適用する商品棚から分かる通り,様々な色の商品が存在するため,画像全体に同一の閾値を設定すると商品ごとの検出精度に差が生じる.この問題に対処するため,画像内座標を利用し,商品ごとに閾値を設定する.\figref{colorgray}に示したグレースケール変換では検出精度が低くなる商品棚において,各商品に設定した閾値を\tabref{colorikiti}に示す.この閾値は,\figref{colorgray}に示したグレースケール変換を適用する商品棚において,存在する商品に対し様々な閾値を試した結果,最も検出精度が高くなると判断した値である.商品ごとの閾値設定のイメージを\figref{HSVikiti}に示す.また,\tabref{colorikiti}に示す閾値のうち,商品名がnoneとなっているのは,\figref{colorgray}において商品が存在しない棚で誤検出が起こらないか確かめるためのものである.

				\vspace{0.5\baselineskip}
		\begin{table}[h]
			\centering
			\caption{HSV画像における閾値設定}
			\begin{tabular}{|c|r|r|r|}
				\hline	商品名	& 色相(H) & 彩度(S) & 明度(V) \\ \hline
				\hline	白玉粒あんベーグル  & 20~35 & 100~255 & 100~255\\
				\hline	クランベリー&クリームチーズベーグル & 20~35 & 100~255 & 100~255\\
				\hline	イチジク&クリームチーズベーグル & 20~35 & 100~255 & 100~255\\
				\hline  明太ポテトベーグル & 10~25 & 100~255 & 20~200\\
				\hline  ビーフカレーベーグル & 10~25 & 100~255 & 20~200\\
				\hline  ベーコンペッパーベーグル & 20~35 & 100~255 & 100~255\\
				\hline  アップルシナモンベーグル & 20~35 & 100~255 & 100~255\\
				\hline  チーズベーグル & 10~25 & 100~255 & 20~200\\
				\hline  none & 10~70 & 100~255 & 100~255\\
				\hline
			\end{tabular}
			\label{colorikiti}
		\end{table}	

		\vspace{0.5\baselineskip}
		\begin{figure}[H]
			\centering
			\includegraphics[width=.6\linewidth]{図2.16.png}
			\caption{HSV画像の閾値設定イメージ}
			\label{HSVikiti}
		\end{figure}

		\subsection{輪郭抽出の手法}
		輪郭抽出を行うにあたり,RETR EXTERNALとCHAIN APPROX SIMPLEという2つの定数を用いる.
		閾値を設定した画像は,各画素において輝度もしくはHSVの値が閾値を満たすか満たさないかのどちらかに区別される.RETR EXTERNALは,閾値を満たす画素の集合のうち,その最外周の輪郭のみを抽出する\cite{2.12}.閾値を満たす画素を白,満たさない画素を黒で表現した商品棚を\figref{sirokuro}に示す.また,RETR EXTERNALによる輪郭抽出のイメージを\figref{retr}に示す.

		\vspace{0.5\baselineskip}
		\begin{figure}[H]
			\centering
			\includegraphics[width=.6\linewidth]{図2.17.png}
			\caption{商品と背景を白と黒で表現した画像}
			\label{sirokuro}
		\end{figure}

		\vspace{0.5\baselineskip}
		\begin{figure}[H]
			\centering
			\includegraphics[width=.7\linewidth]{図2.18.png}
			\caption{RETR EXTERNALの輪郭抽出イメージ}
			\label{retr}
		\end{figure}

		輪郭は無数の点によって表現される.CHAIN APPROX SIMPLEは,輪郭が完全に直線になる部分の点を省いて表現する\cite{2.12}.これによって,リアルタイム性が求められる本研究において,特に重要となる処理の高速化を実現できる.CHAIN APPROX SIMPLEによる輪郭表現のイメージを\figref{chain}に示す.

		\vspace{0.5\baselineskip}
		\begin{figure}[H]
			\centering
			\includegraphics[width=.7\linewidth]{図2.19.png}
			\caption{CHAIN APPROX SIMPLEの輪郭抽出イメージ}
			\label{chain}
		\end{figure}

	\section{類似商品の識別}
	\figref{similar}に示す見た目の近い異なる商品から分かる通り,無人店舗には見た目から種類の識別が困難な商品が存在する.そこで,抽出した輪郭を基に,画像内座標を用いて商品の種類を識別する.

	\vspace{0.5\baselineskip}
	\begin{figure}[H]
		\centering
		\includegraphics[width=.7\linewidth]{図2.20.png}
		\caption{見た目の近い商品}
		\label{similar}
	\end{figure}
	
		\subsection{輪郭の座標表現}
		抽出した商品の輪郭を,1点の座標で表現する.手法としては,輪郭をバウンディングボックスと呼ばれる矩形で表し,その中心点の座標を算出する.\figref{box}に示すバウンディングボックスおよび中心点のイメージから分かる通り,バウンディングボックスは,抽出した輪郭を完全に含み,かつ最小になる矩形であり,バウンディングボックス左上の画像内座標$(x,y)$,バウンディングボックスの幅$w$,バウンディングボックスの高さ$h$という4つの情報と,これらの情報から算出したバウンディングボックスの中心点の画像内座標で表現する.
		\noindent 中心点の画像内$x$座標$C_x$は

		\begin{equation} % equation環境(式番号付き0)
			C_x = x + \frac{w}{2}
			\label{Cx}
		\end{equation}

		\noindent であり,画像内$y$座標$C_y$は

		\begin{equation} % equation環境(式番号付き0)
			C_y = y + \frac{h}{2}
			\label{Cy}
		\end{equation}

		\noindent である.バウンディングボックスで商品の輪郭を表現し,画像内の商品の位置を1つの座標$(C_x,C_y)$で定義することによって,商品の画像内の位置情報を利用する処理の高速化が期待できる.

		\vspace{0.5\baselineskip}
		\begin{figure}[H]
			\centering
			\includegraphics[width=.6\linewidth]{図2.21.png}
			\caption{バウンディングボックスおよび中心点イメージ}
			\label{box}
		\end{figure}

		\figref{sirokuro}に示した抽出した輪郭を白で表現した画像から分かる通り,商品以外に対しても輪郭抽出は行われる.そのため,商品が存在する範囲のみにバウンディングボックスの表現を適用し,商品以外の輪郭の座標データを取得しないよう設定する.これによって,処理の高速化が期待できる.商品が存在する範囲のみでバウンディングボックスを表現するイメージを\figref{chushutujoken}に示す.

		\vspace{0.5\baselineskip}
		\begin{figure}[H]
			\centering
			\includegraphics[width=.7\linewidth]{図2.22.png}
			\caption{商品のみ輪郭抽出のイメージ}
			\label{chushutujoken}
		\end{figure}

		\figref{nosize}に示す輪郭のサイズ指定無しのバウンディングボックス表現から分かる通り,バウンディングボックスは極端に小さい輪郭や大きい輪郭も表現してしまう.そのため,座標データを取得するバウンディングボックスをサイズでフィルタリングする.具体的には,14500~75000画素数(画素面積)のバウンディングボックスのみ座標データを取得するように設定しており,これは様々な条件でフィルタリングを行った結果,極端なサイズの輪郭を排除し,全商品の輪郭を十分に表現できると判断した閾値である.輪郭のサイズによるフィルタリングを行うことで,ノイズを除去し,余計なデータの取得,計算を避け,処理を高速化できる.輪郭のサイズによるフィルタリングを行ったバウンディングボックス表現を\figref{oksize}に示す.

		\vspace{0.5\baselineskip}
		\begin{figure}[H]
			\centering
			\begin{minipage}{.45\linewidth}
				\centering
				\includegraphics[width=\linewidth]{図2.23.png}
				\caption{サイズ指定なしの検出}
				\label{nosize}
			\end{minipage}
			\hfill
			\vspace{0.5\baselineskip}
			\begin{minipage}{.45\linewidth}
				\centering
				\includegraphics[width=\linewidth]{図2.24.png}
				\caption{サイズ指定ありの検出}
				\label{oksize}
			\end{minipage}
		\end{figure}

		\subsection{座標による商品の識別}
		画像処理による検出を実用化する際,特定の距離と画角からの撮影を想定している.また,無人店舗の商品棚は特定の位置に商品が配置されている.これらを踏まえ,各商品の棚の範囲情報と,抽出した商品の輪郭を表現するバウンディングボックスの中心点の座標情報から商品の種類を特定する.
		\figref{shelfwide}に示す商品の範囲情報のイメージから分かる通り,各商品の棚の範囲を矩形(以下棚範囲矩形とする)で表現する.手法としては,棚範囲矩形の左上と右下の画像内座標を事前に取得し,商品ごとに定義する.これは,画像内座標が商品によって異なるためである.また,棚範囲矩形と商品の名称を対応させ,各商品の存在する範囲を定義する.% 画像内座標を取得するコードをソースコード2.1に示す.

		\vspace{0.5\baselineskip}
		\begin{figure}[H]
			\centering
			\includegraphics[width=.7\linewidth]{図2.25.png}
			\caption{棚範囲矩形のイメージ}
			\label{shelfwide}
		\end{figure}

		% \begin{lstlisting}[language=Python, caption=ソースコード, label=source1]
			
		% \end{lstlisting}
		
		商品名を対応させた棚範囲矩形と,2.4.1節で述べた商品の輪郭を表すバウンディングボックスの中心点の座標情報を用いて,商品の種類を特定する.具体的には,バウンディングボックスの中心点が特定の棚範囲矩形内に含まれているか,座標情報を用いることで検出し,含まれていた場合,その棚範囲矩形に対応する商品名を割り当てることで,バウンディングボックスが表す商品を特定する.バウンディングボックスの中心点と棚範囲矩形を用いた商品の特定イメージを\figref{shelfimage}に示す.

		\vspace{0.5\baselineskip}
		\begin{figure}[H]
			\centering
			\includegraphics[width=.7\linewidth]{図2.26.png}
			\caption{棚範囲矩形を用いた商品検出のイメージ}
			\label{shelfimage}
		\end{figure}

	\section{検出結果と考察}
	画像処理による商品検出では,\figref{onigiri}に示したおにぎりなどの白に近い色が特徴的な商品が並ぶ棚に対しては,グレースケール変換を利用し,\figref{colorful}に示した様々な色の商品が並ぶ棚に対しては,HSV変換を利用する.

		\subsection{グレースケール画像の検出結果}
		グレースケール画像の検出結果を\figref{grayresult}に示す.この図では,商品ごとの棚範囲矩形,輪郭のバウンディングボックスとその中心点を表現しており,バウンディングボックスの誤検出を確かめるため,商品が存在しない一部の棚にも閾値100~255の棚範囲矩形を設定している.また,棚範囲矩形をそれぞれ異なる色で表現し,輪郭のバウンディングボックスを対応する棚範囲矩形と同じ色で示している.

		\vspace{0.5\baselineskip}
		\begin{figure}[H]
			\centering
			\includegraphics[width=.6\linewidth]{図2.27.png}
			\caption{グレースケール画像の検出結果}
			\label{grayresult}
		\end{figure}

		\figref{grayresult}の検出結果から分かる通り,輪郭の検出と商品の識別が可能であり,商品が存在しない場所で輪郭のバウンディングボックスの誤検出も発生しないことが分かる.一方で,複数の商品をまとめて1つと検出してしまう現象が確認できる. 

		\subsection{HSV画像の検出結果}
		HSV画像の検出結果を\figref{HSVresult}に示す.2.5.1節で述べたグレースケール画像の検出と同様,棚範囲矩形と輪郭のバウンディングボックス,その中心点,商品が存在しない棚の棚範囲矩形の設定,輪郭のバウンディングボックスと棚範囲矩形の対応の同色表示をそれぞれ表現している.ここで,商品が存在しない棚における棚範囲矩形の閾値は\tabref{colorikiti}に示すnoneの値を用いる.

		\vspace{0.5\baselineskip}
		\begin{figure}[H]
			\centering
			\includegraphics[width=.6\linewidth]{図2.28.png}
			\caption{HSV画像の検出結果}
			\label{HSVresult}
		\end{figure}

		\figref{HSVresult}の検出結果から分かる通り,輪郭の検出と商品の識別が可能であり,商品が存在しない場所で輪郭のバウンディングボックスの誤検出も発生しないことが分かる.また,\figref{colorgray}に示した様々な色の商品が並ぶ棚のグレースケール画像から分かる通り,輝度の値がそれぞれ異なる商品が並ぶ棚は,グレースケール変換を用いた場合,検出精度が低くなるが,HSV変換を用いることで,検出精度の向上が見込める.一方で,グレースケール画像と同様に,複数の商品をまとめて1つと検出してしまう現象が確認できる.

		\subsection{考察}
		2.5.1節および2.5.2節で述べた検出結果から,白に近い色が特徴的な商品が並ぶ棚に関してはグレースケール変換を用い,様々な色の商品が並ぶ棚に関してはHSV変換を用いるのが最適だと考える.\figref{cameraclass}に示す無人店舗の商品棚の分類から分かる通り,グレースケール変換を適用すべき商品棚とHSV変換を適用すべき商品棚は,それぞれ全体の半分の棚範囲を有している.そのため,実用化の際には,グレースケール画像を取得するカメラ,HSV画像を取得するカメラの計2台が必要だと考える.

		\vspace{0.5\baselineskip}
		\begin{figure}[H]
			\centering
			\includegraphics[width=.6\linewidth]{図2.29.png}
			\caption{画像処理を適用する棚分類}
			\label{cameraclass}
		\end{figure}

		\figref{cameraclass}に示す無人店舗の設置場所から分かる通り,無人店舗は屋内に設置されている.そのため,天気をはじめとする外部環境の変化の影響を受けづらく,適切な閾値を設定することで,安定した検出が可能であると考える.しかし,2.5.1節および2.5.2節で述べた検出結果から,グレースケール画像を用いた検出およびHSV画像を用いた検出どちらにおいても,正確な個数の検出は困難だと考える.

		\vspace{0.5\baselineskip}
		\begin{figure}[H]
			\centering
			\includegraphics[width=.5\linewidth]{図2.30.JPG}
			\caption{画像処理を適用する棚分類}
			\label{cameraclass}
		\end{figure}


	
	% --- 超音波センサによるデータ収集(chapter3)
	\graphicspath{{./contents/chapter3/figures/}}	% 図・画像ファイルが保存されたディレクトリまでのパス
\lstset{inputpath=./contents/appendix/source_code/program_files/}

\chapter{超音波センサによる在庫の把握}
	本章では、本研究で行った超音波センサによる在庫把握の手段について述べる。
	\section{超音波センサによる在庫把握の方法}
		本研究では、無人店舗内の商品棚における在庫の把握を超音波センサを用いることによって行った。商品棚の奥側に超音波センサを設置し、センサから発信された超音波が商品に反射して戻ってくるまでの時間を距離に計算し直すことで行う。距離は以下の式で求められる。
		\begin{equation}
			距離 = \frac{音速 \times 時間}{2}
		\end{equation}
		ここで、音速は約343m/sである。計測した距離の長短で在庫の状況を判断する。使用した超音波センサは\figref{ultrasound_sensor}に示すRainbow E-Technology社のHC-SR04である。このセンサは単体では動作しないため、マイコンボードと組み合わせて使用する必要がある。超音波センサで取得したデータをUSB経由でパソコンに送信し、そのデータをパソコンで評価する。

		\vspace{0.5\baselineskip}
		\begin{figure}[h]
			\centering
			\includegraphics{hcsr04.jpg}
			\caption{Rainbow E-Technology社 HC-SR04}
			\label{ultrasound_sensor}
		\end{figure}
	
	\section{研究に使用した装置}
		\subsection{超音波センサの性能}
			HC-SR04の性能を\tabref{HC-SR04}に示す。測定可能距離は0.02m~4.5mであり、商品棚での在庫把握は可能であると考えられる。
			\vspace{0.5\baselineskip}
			\begin{table}[h]
				\centering
				\caption{HC-SR04の精度}
				\begin{tabular}{|c|r|}
					\hline	動作電圧	& 3~5.5V \\
					\hline	測定可能距離	& 0.02~4.5m\\
					\hline	測定方式	& 超音波 \\
					\hline	動作温度	& -10~70℃\\
					\hline
				\end{tabular}
				\label{HC-SR04}
			\end{table}
	
		\subsection{マイコンボードを含めた装置全体}
			前述の通り、この超音波センサは単体では動作しないためマイコンボードと組み合わせる必要がある。マイコンボードにはRaspberry Pi Pico 2 WHを使用した。装置の全体図を\figref{souchi}に示す。
			\vspace{0.5\baselineskip}
			\begin{figure}[h]
				\centering
				\includegraphics[scale=0.1]{souchi1.jpg}
				\caption{装置全体図}
				\label{souchi}
			\end{figure}

	\section{超音波センサの配置}
		超音波センサを商品棚の奥側のスペースに配置し、商品との距離を計測する。


	\section{超音波センサの距離測定に使用するプログラム}
		超音波センサで距離を測定するにあたり、ソースコード\ref{measurepy}に示すPythonのプログラムをマイコンボードに書き込んだ。
		\hypertarget{measurepy}{
			\lstinputlisting[style=mypystyle,caption=マイコンボードに書き込んだPythonプログラム,label=measurepy]{measure.py}
		}
		センサで取得したデータをパソコンで評価するためのプログラムをソースコード\ref{receivepy}に示す。パソコンのOSはWindowsを想定している。
		\hypertarget{receivepy}{
			\lstinputlisting[style=mypystyle,caption=パソコンで実行するPythonプログラム,label=receivepy]{receive_windows.py}
		}

	\section{超音波センサの試行の結果と考察}
		
	
	% --- デプスカメラによるデータ収集(chapter4)
	\graphicspath{{./contents/chapter4/figures/}}	% 図・画像ファイルが保存されたディレクトリまでのパス
\chapter{デプスカメラによるデータ収集}
	\section{使用するデプスカメラの説明}
		本研究ではデプスカメラを用い、センサと商品間の距離を計測した。使用したセンサはである。このセンサ単体では動作しないため、マイコンボードと組み合わせて使用する必要がある。マイコンボードにはRaspberry Pi 4 Model Bを使用した。
	
			\subsection{デプスカメラの精度}
				D435の性能をに示す。測定範囲は0.1m~10mであり、混雑状況のモニタリングの際に観測される距離は測定可能であると考えられる。
				
				\vspace{0.5\baselineskip}
				\begin{table}[h]
					\centering
					\caption{D435の精度}
					\begin{tabular}{|c|r|}
						\hline	動作電圧	& 5V \\
						\hline	消費電流	&	150mA\\
						\hline	視野	&  87°×58°\\
						\hline	測定範囲	&	0.1m~10m\\
						\hline	深度分解能	& 1mm\\
						\hline
					\end{tabular}
					
				\end{table}
	\section{デプスカメラの配置と距離の検出方法}
		デプスカメラを11号館1階の無人店舗の商品棚の奥側のスペースに配置し、商品との距離を計測する。距離の検出方法は、カメラから赤外線を発信し、商品に反射して戻ってくるまでの時間を計測することで行う。距離は以下の式で求められる。


	% --- IoT技術の比較と選択(chapter5)
	\include{./contents/chapter5/IoT_conparion}

	% --- LINE BOTを用いた遠隔在庫把握(chapter6)
	\graphicspath{{./contents/chapter5/figures/}}	% 図・画像ファイルが保存されたディレクトリまでのパス
\chapter{LINE Botを用いた遠隔の在庫把握}
	


	% --- 結論(chapter7)
	\graphicspath{{./contents/chapter7/figures/}}	% 図・画像ファイルが保存されたディレクトリまでのパス
\chapter{総合的考察}
	本章では、第3章で実施した超音波センサによる在庫把握と、第4章で実施したデプスカメラによる在庫把握、、について考察を行う。
	\section{本研究の限界と今後の課題}
	本節では、本研究における限界を明示するとともに、今後の課題について述べる。
		\subsection{本研究の課題}

			\subsubsection{超音波センサに関する課題}
			まず、超音波センサでの在庫数把握において、商品が購入されたあと、在庫が
			一部の小売店では、に

			在庫把握には、装置を大量に用意
	
	\backmatter			% \chapter毎に\phantomsectionの設定が必要(これでハイパーリンクを設定)
	
	% --- 参考文献
	% --- thebibliography環境->preambleで"\usepackage[thebibenv]{thesismacro}"を指定
	\begin{thebibliography}{99} % 二桁分の番号幅を確保
	\setlength{\itemsep}{6pt}	% 文献毎に行間を少し開けるコマンド(不要であればコメントアウト)
		
	% 参考文献定形
	% \bibitem{} ,「」,\url{},最終更新

	% chapter1
	% \bibitem{1.1} SpaceCore,「IoTとは?わかりやすく簡単に解説。仕組みや活用法、課題も紹介」,\url{https://space-core.jp/media/14453}/,最終更新2024.09.17
	\bibitem{1.2} Degital Intelligence チャンネル,「センシングとは? 基礎から注意点を分かりやすく解説し活用例を紹介」,\url{https://www.cloud-for-all.com/blog/what-is-sensing},最終更新2024.12.11
	% \bibitem{1.3} AI総合研究所,「AIとは何か?仕組み・種類・技術用語まで完全ガイド【2025年版】」,\url{https://www.ai-souken.com/article/ai-other},最終更新2025.04.16
	% \bibitem{1.4} Rakuten Mobile,「IoTとは?読み方・意味・仕組みや活用事例を簡単に紹介」,\url{https://network.mobile.rakuten.co.jp/sumakatsu/contents/articles/2025/00458/?msockid=1677875e543f61522301935c554a60a2},最終更新2025.12.10
	% \bibitem{1.5} TTG,「スマートストアとは?仕組みやメリット、国内の事例を徹底解説」,\url{https://ttg.co.jp/media/what-is-smart-store/},最終更新2025.06.10
	\bibitem{1.6} Schoo for Business,「画像処理とは?その特徴や活用される事例について解説する」,\url{https://schoo.jp/biz/column/885},最終更新2025.09.19
	\bibitem{1.7} TTG,「無人店舗の仕組みとは?メリットやデメリットと万引き対策まで徹底解説」,\url{https://ttg.co.jp/media/unmanned-store-system},最終更新2025.05.08

	% chapter2
	\bibitem{2.1} Device HD,「Sony Xperia 10 VI 5G」,\url{https://devicehd.com/smartphones/en/product/66c3181893f2fb776375368c/}
	\bibitem{2.2} SAMURAIENGINNER Blog,「OpenCVとは?できることや特徴をわかりやすく解説」,https://www.sejuku.net/blog/113292,最終更新2025.12.26
	\bibitem{2.3} Qiita,「グレースケール画像のうんちく」,\url{https://qiita.com/yoya/items/96c36b069e74398796f3},最終更新2025.04.20
	\bibitem{2.4} MUSASHI AI,「グレースケール変換」,\url{https://musashi-ai.com/glossary/2023/06/132b15f51be239030f34b40a152b0c724506964c}
	\bibitem{2.5} Mustafa Murat ARAT,「RGB to Grayscale Conversion」,\url{https://mmuratarat.github.io/2020-05-13/rgb_to_grayscale_formulas},最終更新2020.05.13
	\bibitem{2.6} Gigahertz-Optik,「1.6 Spectral Sensitivity of the Human Eye」,\url{https://www.gigahertz-optik.com/en-us/service-and-support/knowledge-base/basics-light-measurement/light-color/spectr-sens-eye/}
	\bibitem{2.7} Britannica,「RGB color model」,\url{https://www.britannica.com/science/RGB-color-model},最終更新2026.01.23
	\bibitem{2.8} PEKO STEP,「HSV色空間」,\url{https://www.peko-step.com/html/hsv.html}
	\bibitem{2.9} OpenCV,「Color spaces in OpenCV」,\url{https://opencv.org/blog/color-spaces-in-opencv/#h-hsv-hue-saturation-value-color-space},最終更新2025.04.29
	\bibitem{2.10} 数理超入門部,「HSV色空間とは?RGBから変換するときの計算式」,\url{https://algorithm.joho.info/image-processing/hsv-color-space/},最終更新2017.07.04
	\bibitem{2.11} IT用語辞典 e-Words,「閾値【threshold】しきい値」,\url{https://e-words.jp/w/%E9%96%BE%E5%80%A4.html}
	\bibitem{2.12} OpenCV オープンソースのすすめ,「OpenCVのfindContours関数を使った画像の輪郭検出」,\url{https://www.argocorp.com/OpenCV/imageprocessing/opencv_find_contours.html}

	% chapter3
	\bibitem{3.1} フェイシングスタンド 河淳 スタンド/仕切板/仕切ワイヤー 【通販モノタロウ】,\url{https://www.monotaro.com/g/02794558/}, 2026/01/29閲覧

	% chapter4
	\bibitem{4.1} ArduCAM/Arducam\_tof\_camera,\url{https://github.com/ArduCAM/Arducam_tof_camera}, 2026/01/29閲覧
	\bibitem{4.2} ToF特集 ToFカメラとは? ToFカメラを使ってできること|inrevium \url{https://www.inrevium.com/pickup/tofcamera/}, 2026/01/29閲覧
	\bibitem{4.3} Arducam ToF Camera SDK – for Raspberry Pi - Arducam Wiki \url{https://docs.arducam.com/Raspberry-Pi-Camera/Tof-camera/Arducam-ToF-Camera-SDK/}
	% chapter5
	\bibitem{5.1} note,「LINE Botとは何か?初心者向けにわかりやすく解説」,\url{https://note.com/bonjour_maman/n/n2a2ce62cde15},最終更新2025.09.16
	\bibitem{5.2} 総務省情報通信政策研究所,「令和6年度情報通信メディアの利用時間と情報行動に関する調査報告書(概要)」p.12,\url{https://www.soumu.go.jp/main_content/001017240.pdf},最終更新2025.06
	\bibitem{5.3} FirstContact,「Messaging APIとは?~意味やできることを解説!~」,\url{https://first-contact.jp/blog/article/messaging-api/},最終更新2025.03.05
	\bibitem{5.4} Qiita,「初心者向け解説:APIとは?その仕組みと活用法を分かりやすく解説」,\url{https://qiita.com/UKI_datascience/items/18605ce56c7d9a4e4ca0},最終更新2025.01.16
	\bibitem{5.5} ミライサーバー,「Flaskとは?基本知識からインストール手順まで詳しく解説!」,\url{https://www.miraiserver.ne.jp/column/about_flask/},最終更新2025.08.18
	\bibitem{5.6} blastengine,「Webhookとは?仕組みやメリット、APIとの違い、利用方法について分かりやすく解説」,\url{https://blastengine.jp/blog_content/webhook/},最終更新2024.10.31
	\bibitem{5.7} Qiita,「Ngrokの使い方・実際に私が使っている事例を紹介」,\url{https://qiita.com/halapolo/items/a9d2345836b0302a264d},最終更新2025.05.14
	\bibitem{5.8} iifx.dev,「Flask開発サーバーをネットワーク公開する完全ガイド」,\url{https://iifx.dev/ja/articles/32624714/flask%E9%96%8B%E7%99%BA%E3%82%B5%E3%83%BC%E3%83%90%E3%83%BC%E3%82%92%E3%83%8D%E3%83%83%E3%83%88%E3%83%AF%E3%83%BC%E3%82%AF%E5%85%AC%E9%96%8B%E3%81%99%E3%82%8B%E5%AE%8C%E5%85%A8%E3%82%AC%E3%82%A4%E3%83%89},最終更新2025.07.19
	\bibitem{5.9} LINE Developers,「Messaging APIの料金」,\url{https://developers.line.biz/ja/docs/messaging-api/pricing/}
	\bibitem{5.10} LINE Developers,「リッチメニューの概要」,\url{https://developers.line.biz/ja/docs/messaging-api/rich-menus-overview/}

	% chapter6

	% chapter7

\end{thebibliography}
	% --- biblatex+biber->preambleで"\usepackage[usebiber]{thesismacro}"&"\addbibresource{.bib}"を指定
	%\printbibliography	% 文献リストの挿入
	
	% --- 謝辞
	\phantomsection\chapter{謝辞}
	本研究にご協力ならびにご助言をいただいた全ての皆様に深く感謝いたします。特に、小川隆申教授ならびに謝文昂助教には、終始熱心なご指導をいただきました。併せて、流体力学研究室の皆様にはご助言とご支援をいただきましたことに深く感謝いたします。
	
	\mainmatter			% \phantomsectionは不要
	% ---------- APPENDIX ----------
	\appendix	% 以下は付録
	% --- 関連プログラム
	% appendixで示すコードののディレクトリパス(絶対パスが使えなくなる & 一度に1つしか指定できないので注意)
\lstset{inputpath=./contents/appendix/source_code/program_files/}
\chapter{プログラム}

	\section{ソースコード}
		以下はソースコードである。超音波センサでの在庫数把握において、マイコンボードで実行したプログラムが\ref{measurepy}である。
		\hypertarget{measurepy}{% ハイパーリンク先の設定\hyperlink{label}{}でここに飛ぶ(内部のlabelと合わせると便利かも)
			\lstinputlisting[style=mypystyle,caption=findContours,label=measurepy]{measure.py}
		}
		超音波センサ
		\ref{machine}に機械学習に用いた一連のプログラムを示す。
		\hypertarget{machine}{% ハイパーリンク先の設定\hyperlink{label}{}でここに飛ぶ(内部のlabelと合わせると便利かも)
			\lstinputlisting[style=mypystyle,caption=MachineLearning,label=machine]{countpeople.py}
		}
	
	% --- 原稿非掲載データ
	\chapter{原稿非掲載データ}
	\section{実験結果}
		例えばここに本文中に載せられなかった実験結果などを載せる.

\end{document}